%%%% Textsatz, Layout, Stil
\usepackage[osf]{mathpazo} % auch Palatino, andere math font
\linespread{1.05}         % Palatino needs more leading (space between lines)
%\usepackage[scaled=0.86]{berasans} % serifenlose Schrift
%\usepackage[scaled]{helvet} serifenlose Schrift
\usepackage[defaultsans]{droidsans} % serifenlose Schrift -- DIE HIER wäre schick!
\usepackage[scaled=1.03]{inconsolata}

%Überschriften serifenlos und über den Rand hängend
\usepackage[sf,sl,outermarks,noindentafter,nobottomtitles]{titlesec}

%könnte alles auch mit \bfseries versehen werden nach Geschmack
\titleformat{\section}[hang]{\LARGE\sffamily}{\thetitle}{8pt}{}
\titleformat{\subsection}[hang]{\Large\sffamily}{\thetitle}{8pt}{}
\titleformat{\subsubsection}[hang]{\large\sffamily}{\thetitle}{8pt}{}
\titleformat{\paragraph}[hang]{\bfseries\sffamily}{\thetitle}{8pt}{}
%Etwas aufwendiger für die Kapitelüberschriften:
\titleformat{\chapter}[display]
{\filleft\Huge\sffamily} %Huge ist die Größe für Titeltext und Nummer
{\fontsize{100pt}{90pt}\selectfont\thechapter}
{-2ex} %is vertical space in [display] mode
%Platz vor dem ganzen Krempel
{\vspace{1ex}}
%Platz danach
[\vspace{1ex}]

%%% TOC design
\usepackage[titles]{tocloft}
\setlength{\cftbeforechapskip}{1ex}
\setlength{\cftbeforesecskip}{0.8ex}
\setlength{\cftbeforesubsecskip}{0.8ex}

%\renewcommand{\cftchapfont}{\sffamily\bfseries}
%\renewcommand{\cftchappagefont}{\sffamily}
%\renewcommand{\cftsecfont}{\sffamily}
%\renewcommand{\cftsecpagefont}{\sffamily}
%\renewcommand{\cftsubsecfont}{\sffamily}
%\renewcommand{\cftsubsecpagefont}{\sffamily}

\renewcommand{\cftpnumalign}{l}
\renewcommand{\cftsecdotsep}{\cftnodots}
\renewcommand{\cftsubsecdotsep}{\cftnodots}
\renewcommand{\cftchapleader}{\hspace{2em}}
\renewcommand{\cftsecleader}{\hspace{2em}}
\renewcommand{\cftsubsecleader}{\hspace{2em}}
\renewcommand{\cftchapafterpnum}{\cftparfillskip}
\renewcommand{\cftsecafterpnum}{\cftparfillskip}
\renewcommand{\cftsubsecafterpnum}{\cftparfillskip}

%%%% Grafiken, Abbildungen
\floatsetup{
    capposition=beside,
    capbesideposition={top,outside},
    facing=yes,
    floatwidth=.75\linewidth,
    capbesidewidth=sidefil,
    capbesidesep=quad,
    floatrowsep=quad,
    %framestyle=colorbox,framearound=all,colorframeset=imgbg,frameset={\fboxrule0pt},
    }
\floatsetup[widefigure]{%
    floatwidth=0.95\textwidth,
    %margins=hangoutside,
    %capposition=beside,
    capposition=below,
    %capbesideposition={top,outside},
    %capbesidewidth=\marginparwidth,
    %capbesideframe=yes,
    %capbesidesep=columnsep,
    %floatrowsep=columnsep,
    %heightadjust=nocaption,
    facing=yes,
    }
\floatsetup[widetable]{%
    floatwidth=0.95\textwidth,
    %margins=hangoutside,
    %capposition=beside,
    capposition=above,
    %capbesideposition={top,outside},
    %capbesidewidth=\marginparwidth,
    %capbesideframe=yes,
    %capbesidesep=columnsep,
    %floatrowsep=columnsep,
    %heightadjust=nocaption,
    facing=yes,
    }
\floatsetup[widefloat]{%
    floatwidth=0.95\textwidth,
    %margins=hangoutside,
    %capposition=beside,
    capposition=below,
    %capbesideposition={top,outside},
    %capbesidewidth=\marginparwidth,
    %capbesideframe=yes,
    %capbesidesep=columnsep,
    %floatrowsep=columnsep,
    %heightadjust=nocaption,
    facing=yes,
    }
\usepackage[]{caption}
\DeclareCaptionLabelSeparator{vbar}{ | } % custom; standard z.B. colon, period, ...
\captionsetup{labelfont=bf,font={sf,footnotesize},format=plain,labelsep=vbar}

