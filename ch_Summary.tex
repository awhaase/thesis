\chapter{Summary} \label{ch_summary}
%As the layer thicknesses enter the sub-nanometer regime for the Cr/Sc multilayer mirrors designed for the water window regime, the model is extended to deliver a consistent result taking into account all of the aforementioned methods. As a result, those are assessed with respect to their applicability and necessity for the structural investigation of such samples.


This thesis has treated the characterization Mo/Si and Cr/Sc multilayer mirror systems by means of the combination of several indirect methods based on reflection, fluorescence and scattering using \gls{euv} and X-ray radiation. Its focus was to validate and improve the applied theoretical models and determine the experimental techniques required to achieve a unambiguous solution to the inverse problem. For the reconstruction of the layer systems structure, a \glsdesc{pso} was applied to fit the model parameters to the measured data from \glsdesc{euv} reflectivity, \glsdesc{xrr}, \glsdesc{reuv} and \glsdesc{xrf} experiments. A \glsdesc{mcmc} algorithm was further employed to deduct the maximum likelihood distribution and thereby confidence intervals based on the measurement and model uncertainties. It was found, that different methods and models had to be applied depending on the system under investigation. The values and confidence intervals determined for each parameter of the respective model, allowed to draw conclusions on the actual structural layout of the samples.

%First, a methodology for a profound analysis on the validity and accuracy of the structural characterization obtained from \glsdesc{euv} reflectivity, \glsdesc{xrr}, \glsdesc{reuv} and \glsdesc{xrf} experiments was presented and applied to different multilayer mirror samples. The uniqueness and accuracy of a solution to the inverse problem is an important aspect to justify the relation of the model parameters to actual sample properties. Thus, for the reconstruction of the layer systems structure, a \glsdesc{pso} was applied to fit the model parameters to the measured data and a \glsdesc{mcmc} algorithm was employed to deduct the confidence intervals based on the measurement and model uncertainties addressing this problem. Depending on the sample properties, different methods have to be combined to improve the accuracy of the result and thereby provide the desired significance or to deliver a consistent solution in the first place. In particular, the different layer thicknesses of the two sample types under investigation in this thesis, the Mo/Si mirrors designed for \nm{13.5} wavelength and the Cr/Sc mirrors for the water window, do have a significant influence on the applicability of the model and the analytic experiments required to deduct a consistent result.

The structural characterization methods could yield layer thicknesses, densities and even the distortion of the interfaces became accessible. However, they lack in the ability to identify these distortions as either roughness or intermixing. This could only be achieved by combining the results of the structural characterization with a method sensitive to roughness and again validate the result with respect to accuracy. This issue was approached by establishing the analysis of \gls{euv} diffuse scattering with radiation impinging near-normal incidence as a suitable technique to deliver this distinction method. The method was introduced by analyzing the state-of-the-art Mo/B$_4$C/Si/C mirror reaching \SI{68.5 \pm 0.7}{\percent} peak reflectance at its operation wavelengths of \nm{13.5}. It was revealed that the high quality, and thus reflectivity, of the sample causes resonant enhancement of diffusely scattered radiation within the stack to significantly contribute to the diffuse scattering intensities. These dynamic effects required consideration in the analysis by employing the theoretical framework of the \glsdesc{dwba} including multiple reflections at the interfaces of the multilayer. With this approach, the roughness properties for the samples could be extracted consistently. By comparing and combining the results of the structural characterization and the roughness analysis a consistent characterization of the multilayer mirrors could be achieved. Thereby, the analysis in this thesis could explain the lack of peak reflectivity compared to the theoretical expectation for an ideal system for both sample systems.

In the unpolished and polished set of the Mo/Si/C multilayer mirrors, it was revealed that the combination of \gls{euv} reflectivity and \glsdesc{xrr} yields an unambiguous result for the molybdenum layer thickness confirming the nominal trend in both sets. The confidence intervals for the molybdenum thickness could be determined ranging from $\nm{0.43}$ down to $\nm{0.24}$, depending on the sample. In comparison, the analysis of \gls{euv} reflectivity alone for the comparison sample Mo/B$_4$C/Si/C only yielded a confidence interval of approximately $\nm{1}$, demonstrating the need for combining multiple datasets despite an excellent agreement of the calculated and measured curves as multiple solutions exist.
The sum of thicknesses of all layers in a period shows a distinct increase for both sets at a certain molybdenum layer thickness associated with a dip in peak reflectance with respect to the theoretical expectation. This effect, while observed in both sets, happens at significantly different molybdenum thicknesses comparing the unpolished with the polished samples.

The analysis of the diffuse scattering intensity allowed an assessment of the interface morphology for those selected samples. The comparison with the structural analysis revealed an increase of roughness associated with the jump in the period thickness and the dip in peak reflectance, which is compensated again at even larger thicknesses. At this point, it may be concluded that these effects are caused by the onset of crystallization in the molybdenum layer causing increased interface disturbances through roughness. In the analysis of the ion polished set, this threshold was shown to have moved towards lower molybdenum thicknesses. This is beneficial to the reflectance at the optimum molybdenum ratio with respect to the rest of the layers in a period, which in the polished set is now unaffected trough roughening due to crystallization. Nevertheless, comparing the roughness values found in the diffuse scattering analysis with the N{\'e}vot-Croce factor from the optimized layer structure model, it became clear that while overall roughness was reduced significantly and led to a significant increase of the reflectivity in the polished set, the Nev{\'o}t-Croce parameter only reduced slightly indicating that intermixing is still largely responsible for the remaining gap to the theoretically achievable reflectivity.

In the case of the Cr/Sc multilayers for the water window spectral range, nominal layer thicknesses within a bilayer period are between \nm{0.7} and \nm{0.8} and thus noticeably thinner than for the Mo/Si systems. It was shown, that an approach to the structural characterization based on a discrete layer model for the chromium and scandium layers does not yield a solution to both, the \gls{euv} reflectivity and \gls{xrr} experiments, with the same set of parameters. Thus, a solution fitting the \gls{euv} reflectivity experiment fails to describe the \gls{xrr} curve and vice versa. The discrete layer model is not suitable to describe the physical structure of the sample sufficiently and any solution found for either one of the experiments can not be related to the physical properties of the sample. Instead, a model describing a gradual interface profile and layers composed of a mixture of both materials was introduced. Based on this gradual model, the intermixing and roughness were parametrized separately and asymmetric interface profiles could be described explicitly.

It was found through the uniqueness and accuracy analysis, that the increased variability of the improved model requires more complementary information by performing \gls{euv} reflectivity, \gls{reuv}, \gls{xrr} and \gls{xrf} experiments to achieve the goal of unambiguity of the solutions. Confidence intervals were determined, by evaluating each dataset individually and by combining all in a single analysis. The solutions and confidence intervals determined prove, that only the combination of all datasets could yield a consistent result. It was found, that none of the regions within the Cr/Sc stack are pure chromium or scandium. Furthermore, the interface regions show a strong asymmetry, which could not be determined with the required significance by any of the analytic experiments alone. For the same reason as stated above, not even the combined analysis could distinguish roughness and intermixing. Those two parameters were shown to have a strong correlation. To distinguish roughness and intermixing, the \gls{euv} diffuse scattering was measured and analyzed similarly as for the Mo/Si samples. The result shows a roughness value of $\sigma_r = 0.17  (-0.01/+0.02)$ nm. Consequently, the intermixing could be determined to be $47 (-4/+3)\, \%$, leaving any of the nominal chromium or scandium layers of the stack to contain parts of the other material. In conclusion, the roughness determined agrees with the roughness observed for the polished Mo/Si/C samples. There, this roughness amplitude still allowed reflectivities much closer to the theoretical maximum value, identifying intermixing as the main cause for the small reflectivity of the Cr/Sc multilayer mirrors for the water window.

In summary, the work presented in this thesis highlights the importance of assessing the uniqueness and accuracy of indirect characterization methods to deduct a meaningful result. As shown on several occasions in the analysis of the multilayer mirrors, reconstructions in very good agreement with the data curves were proven to show ambiguities and inconsistencies, when adding complementary information from other experiments or even when analyzing the data of a single experiment using global optimization algorithms. With the approach of combining multiple analytic techniques and determining confidence intervals of the reconstructed parameters shown here, conclusions on the physical properties of the samples could be drawn reliably. This thesis thus augments the existing characterization methods for multilayer mirrors in that respect. Finally, with the inclusion of \gls{euv} diffuse scattering, a technique to assess the interface morphology was established, suitable for at-wavelength characterization near-normal incidence offering an alternative to grazing-incidence methods such as \gls{gisaxs}. This has some unique advantages as any measurement using small incidence angles is inherently limited to flat or convex surfaces. Focusing mirrors, however, usually are concavely curved and thus not suitable for characterization techniques with grazing angles of incidence. Instead, with \gls{euv} diffuse scattering with radiation impinging near normal incidence, it is possible to extract the roughness information for those samples as well. In addition, the wavelengths of operation for these mirrors are directly applicable to conduct this analysis.

As an outlook extending the scope of this work, it would be interesting to evaluate the gain in accuracy and uniqueness of the solutions by applying the compilation of techniques used for the Cr/Sc system, also to the two Mo/Si/C sample sets. This may prove to be beneficial to further reduce the confidence intervals on the results, most prominently on the thickness of the barrier layers. In particular, as a straight forward approach, the improved model for the Cr/Sc mirrors could be carried over to these systems to investigate the role of the barrier and compound layers in the crystallization based on validated reconstruction parameters and augment the analysis conducted on similar systems elsewhere \cite{bajt_investigation_2001}. In general, it may enhance the accuracy of the analysis to evaluate if including even further methods, e.g.~ellipsometry, could deliver additional complementary information, in particular, on the optical constants of the various materials in the multilayer stack. This may be of interest for novel material combinations for multilayer mirrors designed to operate in other spectral ranges, where no data exists so far, but also for the systems investigated here.