\chapter{Summary} \label{ch_summary}
% \begin{itemize}
%  \item Model uniqueness
%  \item uncertainties
%  \item Distinction between roughness and interdiffusion
%  \item crystallization barrier, main reason is intermixing still
%  \item Cr/Sc systems, asymmetry only by combination of all methods
%  \item Cr/Sc intermixing huge problem
% \end{itemize}
This thesis treats the characterization of multilayer mirror systems by means of the combination of several indirect methods based on reflection, fluorescence and scattering using \gls{euv} and X-ray radiation. In chapter~\ref{ch_spec}, a methodology for a profound analysis on the validity and accuracy of the structural characterization obtained from \glsdesc{euv} reflectivity, \glsdesc{xrr}, \glsdesc{reuv} and \glsdesc{xrf} experiments was presented and applied to different multilayer mirror samples. The uniqueness and accuracy of a solution to the inverse problem is an important aspect to justify the relation of the model parameters to actual sample properties. Thus, for the reconstruction of the layer systems structure, a \glsdesc{pso} was applied to fit the model parameters to the measured data and a \glsdesc{mcmc} algorithm was employed to deduct the confidence intervals based on the measurement and model uncertainties addressing this problem. Depending on the sample properties, different methods have to be combined to improve the accuracy of the result and thereby provide the desired significance or to deliver a consistent solution in the first place. In particular, the different layer thicknesses of the two sample types under investigation in this thesis, the Mo/Si mirrors designed for \nm{13.5} wavelength and the Cr/Sc mirrors for the water window, do have a significant influence on the applicability of the model and the analytic experiments required to deduct a consistent result.

Starting from the results for the structural properties of the layer systems, the characterization of the interface morphology of the samples was addressed in chapter~\ref{ch_diff}. While delivering a result on the influence of both, the effect of roughness and intermixing at the interfaces in combination through the N{\'e}vot-Croce parameter, none of the analytic experiments conducted for the structural reconstruction can distinguish those two effects. This issue was approached by establishing the analysis of \gls{euv} diffuse scattering with radiation impinging near-normal incidence as a suitable technique to deliver this distinction method. Sec.~\ref{ch_diff:sec_PTB17} introduces the method by analyzing a state-of-the-art Mo/B$_4$C/Si/C sample reaching \SI{68.5 \pm 0.7}{\percent} peak reflectance at its operation wavelengths of \nm{13.5}. It was revealed that the high quality, and thus reflectivity, of the sample causes resonant enhancement of diffusely scattered radiation within the stack to significantly contribute to the diffuse scattering intensities. These dynamic effects require to be considered in the analysis by employing the theoretical framework of the \glsdesc{dwba} including multiple reflections at the interfaces of the multilayer. With this approach, the roughness properties for the samples could be extracted consistently.

In conclusion, comparing and combining the results of the structural characterization from chapter~\ref{ch_spec} and the roughness analysis in chapter~\ref{ch_diff} delivers a consistent characterization of the multilayer mirrors. The systems investigated in this thesis can be attributed to either the mirrors designed for \nm{13.5} wavelength made from molybdenum and silicon or the water window spectral range using chromium and scandium. As outlined in the introduction, the scope of this thesis is to yield an explanation for the lack of peak reflectivity compared to the theoretical threshold.

In Sec.~\ref{ch_spec:sec_mo_si_c} and Sec.~\ref{ch_diff:sec_mo_si_c}, two sets of Mo/Si/C multilayer samples were investigated. In both sets, the thickness of the molybdenum layer was increased from sample to sample from nominally \nm{1.7} to \nm{3.05} crossing the threshold for crystallites forming in these layers. Furthermore, the second set was treated using an ion polishing technique during deposition with the goal to reduce roughness at the interfaces. Sec.~\ref{ch_spec:sec_mo_si_c} reveals that the combination of \gls{euv} reflectivity and \glsdesc{xrr} yields an unambiguous result for the molybdenum layer thickness following the nominal trend in both sets. The confidence intervals for the molybdenum thickness could be determined ranging from $\nm{0.43}$ down to $\nm{0.24}$, depending on the sample. In comparison, the analysis of \gls{euv} reflectivity alone for the comparison sample Mo/B$_4$C/Si/C in Sec.~\ref{ch_spec:sec_PTB17} only yielded a confidence interval of approximately $\nm{1}$, demonstrating the need for combining multiple datasets despite an excellent agreement of the calculated and measured curves as multiple solutions exist.

The sum of thicknesses of all layers in a period shows a distinct increase for both sets at a certain molybdenum layer thickness associated with a dip in peak reflectance with respect to the theoretical expectation as illustrated in Fig.~\ref{ch_spec:fig_EUV_peak_refl}. This effect, while observed in both sets, happens at significantly different molybdenum thicknesses comparing the unpolished with the polished samples. For those samples, the diffuse scattering intensity was analyzed in Sec.~\ref{ch_diff:sec_PTB17} to assess the interface morphology. The comparison with the structural analysis revealed an increase of roughness associated with the jump in the period thickness and the dip in peak reflectance, which is compensated again at even larger thicknesses. At this point, it may be concluded that these effects are caused by the crystallites forming in the molybdenum layer causing increased interface disturbances. By applying the ion polishing technique during the deposition of the sample, this threshold was moved towards lower molybdenum thicknesses. This is beneficial to the reflectance at the optimum molybdenum ratio with respect to the rest of the period, which in the polished set is now unaffected by roughening due to crystallization. However, comparing the roughness values found in the diffuse scattering analysis with the N{\'e}vot-Croce factor from the optimized layer structure model, it becomes clear that while overall roughness was reduced significantly in the polished set, the Nev{\'o}t-Croce parameter only reduced slightly indicating that intermixing is largely responsible for the remaining gap to the theoretically achievable reflectivity and thus requires to be addressed for even further improvement of the mirror.

In the case of the Cr/Sc multilayers for the water window spectral range, nominal layer thicknesses within a bilayer period are between \nm{0.7} and \nm{0.8} and thus noticeably thinner than for the Mo/Si systems. It was shown in Sec.~\ref{ch_spec:sec_CrSc_resconstrution_binary}, that an approach to the structural characterization based on a discrete layer model for the chromium and scandium layers does not yield a solution to both, the \gls{euv} reflectivity and \gls{xrr} experiments, with the same set of parameters. Thus, a solution fitting the \gls{euv} reflectivity experiment fails to describe the \gls{xrr} curve and vice verse. The discrete layer model is thus not suitable to describe the physical structure of the sample sufficiently and any solution to either one of the experiments and the results can not be related to the physical properties of the sample. Instead, a model describing a gradual interface profile and layers composed of a mixture of both materials was introduced in Sec.~\ref{ch_spec:sec_CrSc_gradual_model}. Based on this gradual model, the intermixing and roughness were parametrized separately and asymmetric interface profiles could be described explicitly.

For this system the scope was to consistently determine all parameters of the improved model and thereby identify the cause for the low reflectivity. The question of uniqueness and accuracy of the solution was therefore addressed by performing \gls{euv} reflectivity, \gls{reuv}, \gls{xrr} and \gls{xrf} experiments. Those were analyzed with respect to their utility to deliver solutions for that model in Sec.~\ref{ch_spec:sec_CrSc_results}, by evaluating each dataset individually and by combining all in a single analysis. The solutions and confidence intervals determined prove, that only the combination of all datasets could yield a consistent result. It was found, that none of the regions within the Cr/Sc stack are pure chromium or scandium. Furthermore, the interface regions show a strong asymmetry, which could not be determined with the required significance by any of the analytic experiments alone. For the same reason as stated above, not even the combined analysis could distinguish roughness and intermixing. Those two parameters were shown to have a strong correlation requiring either one to exists in the sample. Thus identifying these as the explanation for the low reflectivity as compared to the theoretically possible values. To distinguish roughness and intermixing, the \gls{euv} diffuse scattering was measured and analyzed in Sec.~\ref{ch_diff:sec_CrSc}. The result shows a roughness value of $\sigma_r = 0.17  (-0.01/+0.02)$ nm. Consequently, the intermixing could be determined to be $47 (-4/+3)\, \%$, leaving any of the nominal chromium or scandium layers of the stack to contain at least that amount of the other material on average. In conclusion, the roughness determined agrees with the roughness observed for the polished Mo/Si/C samples, leaving intermixing as the main cause for the small reflectivity of the Cr/Sc multilayer mirrors for the water window and thus the point were the improvement of the fabrication process should be made.


%In the first category, two sample sets of of Mo/Si/C multilayer mirrors were investigated. In both sets, the thickness of the molybdenum layer was increased from sample to sample from nominally \nm{1.7} to \nm{3.05} and thus crossing the threshold for crystallites forming in these layers. 

% The characterization methodology was applied to two sets of Mo/Si/C multilayer mirrors for the wavelength of \nm{13.5}. In both sets, the thickness of the molybdenum layer was increased from sample to sample from nominally \nm{1.7} to \nm{3.05} and thus crossing the threshold for crystallites forming in these layers. It could be shown, that the combination of \gls{euv} reflectivity and \glsdesc{xrr} demonstrated in Sec.~\ref{ch_spec:sec_mo_si_c} yielded an unambiguous result for the molybdenum layer thickness following the anticipated trend. The confidence intervals for the molybdenum thickness could be determined ranging from $\nm{0.43}$ down to $\nm{0.24}$, depending on the sample. An analysis of a \gls{euv} reflectivity experiment alone for a comparison sample could only yield an accuracy of approximately $\nm{1}$, as shown in detail in Sec.~\ref{ch_spec:sec_PTB17}.
% 
% The thesis evolves around two major sample types distinguished through their application as mirrors for different spectral ranges. With the semiconductor industry moving towards next-generation lithography systems employing radiation at \nm{13.5}, multilayer mirrors made from Mo/Si layers are investigated. First a high-reflective mirror reaching \SI{68.5 \pm 0.7}{\percent} peak reflectance at its operation wavelengths of \nm{13.5} and \SI{6.75}{\degree} angle of incidence from the surface normal was investigated to prove the methodologic approaches. It consists of a periodic Mo/B$_4$C/Si/C stack which includes boroncarbite and carbon as barrier layers to prevent compound formation and interdiffusion. Secondly, two sample series made from periodic Mo/Si/C stacks with varying molybdenum thickness from sample to sample were characterized. In each series, the molybdenum layer thickness was varied nominally from \nm{1.7} to \nm{3.05}, thereby crossing the optimum ratio to the silicon layer thickness for maximum reflectance as mirrors for \nm{13.5} wavelength and crossing the threshold for crystallites forming in those layers. With an additional interface polishing treatment applied to only one of the two sets, this system posed an interesting challenge to the characterization conducted. Finally, multilayer mirrors suitable to reflect radiation in the water window spectral range were studied. More specifically, those multilayer systems serve as mirrors for radiation close to the Sc L-edge at approximately \nm{3.1} impinging at only \SI{1.5}{\degree} from the surface normal. For this, the material combination of chromium and scandium has to form alternating layers with thicknesses in the sub nanometer regime.
% 
% In the first part of this thesis, it has been demonstrated that the problem of uniqueness and accuracy associated with the structural reconstruction of multilayer systems based on measurements of \gls{euv} reflectivity, \gls{xrr}, \gls{reuv} and \gls{xrf} is of high relevance to deduct reliable reconstructions. For that purpose, the experimental data was analyzed using global optimization algorithms. More specifically, \gls{pso} optimization, yielded parameter values for the model, which were evaluated globally within predefined bounds. The results deliver a model, which yields a calculated curve that resembles the measured data of the respective experiment. While fitting a model to the measured data is a standard method and each of the characterization techniques applied here have been used before in relation to multilayer systems, this thesis approaches the topic of the validity of a solution found by employing the \gls{mcmc} maximum likelihood estimation. Based on this, confidence intervals could be obtained for every optimization result taking into account experimental and model uncertainties. With this result a judgment on the significance on the solution became possible.
% 
% The study of the structural properties of the Mo/B$_4$C/Si/C multilayer sample using only \gls{euv} reflectivity resulted in a very good agreement of the theoretical curve and the measured data. However, several solutions with a very similar quality could be deducted showing significantly different model parameters. In the second step, by implementing the \gls{mcmc} approach, the confindence intervals of each of the parameters confirmed the large ambiguity with respect to the individual layer thicknesses found. In conclusion, the layer thickness, e.g.~of the molybdnum layer, could only be determined to be $d_\text{Mo} = 3.137 ({-0.587}/{+0.560})$ nm, and thus only at an total accuracy of one nanometer.
% 
% To improve that accuracy with respect to the structural reconstruction, for the two series of Mo/Si/C samples, which are comparable to the Mo/B$_4$C/Si/C system, additional \gls{xrr} data was taken into account in the optimization procedure and analyzed using a combined functional. It could be proven that a significant improvement through smaller confidence intervals for the resulting model parameters could be obtained compared to the analysis based on the \gls{euv} reflectivity data only. Most prominently, the linear increase of the molybdenum thickness from sample to sample in both series was resembled in the reconstruction. The confidence intervals for the molybdenum thickness could be reduced to total accuracies ranging from $\nm{0.43}$ down to $\nm{0.24}$ in some cases. Due to the improved accuracy, the results allowed the interpretation of a jump of the total period thickness significantly larger than the respective confidence interval for this parameter. It was observed in both series at certain molybdenum layer thicknesses and could be associated with the threshold of molybdenum forming crystallites at certain thicknesses in accordance with the literature and a decrease of the peak reflectance measured in the experiments.

% \section{Kiessig}
% We have applied near-normal incidence diffuse scattering in the EUV spectral range to analyze the interfacial roughness of Mo/Si multilayers. At-wavelength reciprocal space maps in the vicinity of the main Bragg resonance of the multilayer were recorded for the first time via angle and wavelength resolved scatterometry. We observed intensity enhancements in the off-specular scattering. Experiments in different geometries revealed a dependence of the off-specular scattered intensity on the measurement geometry.
% 
% Numerical simulations based on the distorted-wave Born approximation (DWBA) have been performed. The comparison of semi-kinematic simulations with dynamic calculations show that dynamic effects, i.e.~multiple reflections at the interfaces, cannot be neglected. The semi-kinematic approach is invalid when either incidence or exit angle fulfill the Bragg condition. In addition, dynamic multiple reflections caused by increased reflectivity due to the Kiessig fringes close to the main Bragg resonance contribute significantly to the off-specular scattering distribution. The simulations show that the limited bandpass reflection property of the multilayer causes the geometry-dependent diffuse scattering in conjunction with the dynamic maxima.
% 
% Therefore, in the determination of the interface morphology from co-planar reciprocal space maps a multilayer enhancement factor has to be considered to extract the power spectral density (PSD). We have applied our model to two different measurement geometries with two different angles of incidence for the specular case. Together with the multilayer composition determined from modeled specular reflectivity curves rigorous simulations of the diffuse scattering intensity caused by the multilayer were possible, in excellent agreement with the measured data. The average lateral power spectral density could then be extracted with regard to the multilayer enhancement factor equivalently for any measurement geometry. In addition, measurements along the $q_z$ direction provide information on the vertical correlation of interfaces, i.e.~the determination of the vertical correlation length. 
% 
% In conclusion, the consideration of the dynamic effects in the DWBA allows the characterization of the multilayer with respect to its roughness properties. The diffuse scattering measurements corrected for the multilayer enhancement factor provide a measure of the power spectral density. Thus, this method is not restricted to the specific representation of the power spectral density used in our model. Alternative power spectral density models have been discussed in the literature \cite{PhysRevB.38.2297, PhysRevB.48.2873} and are equivalently applicable in the numerical simulations.
% 
% \section{Mo/Si}
% We have demonstrated the analysis of Mo/Si/C multilayer systems designed for normal incidence operation in the EUV spectral range. Two sample sets were prepared with a varying Mo layer thickness, while keeping the total period thickness approximately constant. For the second set an interface polishing treatment was applied during deposition of each period to decrease the interface roughness. We have analyzed the thickness of the individual layers in specular EUV and X-ray reflectivity measurements. We observed the designed increase in the Mo content in good agreement with the nominal values. We also observed for both sample sets a simultaneous jump in the total period thickness $D$ and the Mo layer thickness around $d_\text{Mo} = 2.5$ nm for the unpolished samples and $d_\text{Mo} = 2.2$ nm for the polished samples. In comparison to the suspected trend of the peak reflectance with $d_\text{Mo}$ we observed two samples with lower reflectance in both sets, one exactly at the position of this jump and the other sample with nominally $0.15$ nm lower $d_\text{Mo}$. Furthermore, the evaluation of the diffuse scatter revealed increased roughness throughout the ML stack for the samples just at the thickness jump. At least for the unpolished samples, this higher roughness is not observed from evaluating the specular reflectance alone, where the reflectance is diminished by the combined effects of roughness, interdiffusion and compound formation, which is represented by an effective $\sigma$-value in the N\'{e}vot-Croce factor. For the polished samples, the overall effect is smaller and the enhanced scatter is also observed in the total N\'{e}vot-Croce damping factor. The roughness amplitudes as derived from the diffuse scatter, however, have much smaller confidence intervals.
% 
% We interpret our findings in line with the observation of the formation of crystallites in the Mo layer \cite{bajt_investigation_2001} at around $2$ nm thickness. Particularly, we assign the threshold to the lower thickness where the reflectance first drops without an observation of increased roughness by diffuse scatter. This is explained (analog to \cite{bajt_investigation_2001}) by the crystallization  process starting with increased interdiffusion and small seeds corresponding to a short correlation length, yielding high spacial frequency roughness, not correlated throughout the stack. The corresponding scatter is thus not resonantly enhanced. Without the enhancement, it is below the detection threshold of our experiment. With increasing crystallites, the diffuse scatter becomes observable at slightly higher Mo thickness. Note that for the unpolished sample, the threshold coincides with the point where the ideal Mo-to-Si ratio should yield the highest reflectance in agreement with the findings in \cite{bajt_investigation_2001}. For the polished samples, this threshold is shifted to thinner Mo layers around $d_\text{Mo} = 1.77(-0.22/+0.19)$ nm. This is beneficial for the peak reflectance, which is higher at the optimum ratio, than for the unpolished set. In both cases, a smoothening occurs for even larger Mo thickness, restoring the roughness to its value below the threshold. The evaluation of the diffuse scatter shows an overall lower roughness for the polished samples and, particularly, a destruction of vertical roughness correlation throughout the stack and an increase of the in-planar correlation length, as intended by the polishing. 
% 
% Finally, we note that the analysis methods applied here allow to consistently determine the Mo layer thickness and the average power spectral density roughness for the interfaces throughout the full multilayer stack. The application of these methods to Mo/Si multilayer samples with varying Mo thickness with/without polishing illustrated the power of the method for the investigation of structural changes and confirmed previous findings on the onset of Mo crystallization.
% 
% \section{Cr/Sc}
% In conclusion, we have demonstrated a robust method to characterize ultra-thin 
% multilayer systems with subnanometer layer thicknesses unambiguously. Layer 
% thicknesses in the subnanometer region are necessary for near-normal incidence 
% reflective mirrors in the water window spectral range. However, they come with 
% the cost of increasing susceptibility to disturbances in the interfaces at the 
% layer boundaries. This limits the achievable reflectance to values well below 
% the theoretical threshold, posing a demand for ideally non-destructive 
% characterization methods. The main mechanisms for diminished reflectance are 
% interdiffusion and roughness. With these effects ranging on the order of the 
% layer thickness, models based on binary layer stacks become inadequate to 
% describe the physical situation. In order to find a proper representation of 
% the multilayer sample, more sophisticated models with an explicit description 
% of the gradual interdiffusion layers become necessary. This inevitably 
% increases the number of parameters to be determined in analytical experiments. 
% Finding an unambiguous solution is challenging and can only be achieved with a 
% combined analysis of several non-destructive techniques.
% 
% We performed a rigorous analysis of several experimental methods to determine 
% the model parameters representing one Cr/Sc sample. The optimal set of 
% parameters was determined by applying a particle swarm optimizer in conjunction 
% with a Markov-chain Monte Carlo method to verify the uniqueness of the solution 
% and derive confidence intervals for all parameters in all experiments. The set 
% of analytical methods we employed were EUV and X-ray reflectance, resonant EUV 
% reflectance across the Sc L edge as well as X-ray standing wave fluorescence at 
% the Sc-K and Cr-K lines across the first Bragg peak. The analysis of each 
% method shows different sensitivities for specific parameters of the model. The 
% EUV reflectance shows sensitivity for the optical contrast, i.e. the 
% intermixing $\eta$ and the roughness $\sigma_r$. With the resonant EUV 
% reflectance this is further improved and additional sensitivity is added with 
% respect to the ratio of Sc and Cr as well as the total period thickness $D$. 
% The XRR measurement, on the other hand, yields better confidence intervals for 
% the roughness $\sigma_r$ due to the appearance of the second Bragg peak.
% Finally, GIXRF delivers a method to resolve the 
% multilayer structure spatially and thus the interdiffusion layer thickness 
% $\sigma_d$ and the Sc to Cr ratio.
% 
% Within the verified confidence intervals the MCMC methods reveal a remaining 
% correlation between the intermixing parameter and the roughness factor, which 
% could not be resolved with the experiments in specular geometry. We therefore 
% performed a measurement of the off-specular diffuse scattering to distinguish 
% between the roughness and the interdiffusion. The results of these analyses 
% reveal a high degree of roughness correlation throughout the multilayer with 
% interface roughness values comparable with the best fit obtained in the 
% combined analysis. With the combination of all these methods, a robust result 
% could be derived with improved confidence intervals. Most notably, only the 
% combined analysis can detect the asymmetry of the interdiffusion layers 
% $\Gamma_\sigma$. It should also be noted here that the interdiffusion width 
% $s_d$ is much larger than the roughness values $\sigma_r$. Also none of the 
% layers was found to have the index of refraction of pure Cr or Sc, 
% respectively. This is reflected through the non-vanishing intermixing parameter 
% $\eta>0$. Thus, it can be concluded that while roughness still exists, 
% intermixing and interdiffusion are the main cause of diminished reflectance for 
% the Cr/Sc multilayer system studied here.