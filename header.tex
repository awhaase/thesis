%%%% common header file for final and draft mode
\usepackage[utf8]{inputenc}
\usepackage[T1]{fontenc}
%\usepackage{lmodern}
\usepackage[ngerman,english]{babel}
\usepackage[protrusion, expansion]{microtype}
\usepackage{textcomp}
\usepackage{xspace}
\usepackage[dvipsnames,svgnames]{xcolor}
\usepackage{floatrow}
\DeclareColorBox{imgbg}{\fcolorbox{white}{white}}

\ifdraft{%
    \linespread{1.25}  % Zeilenabstand fuer Korrekturen

\usepackage{geometry}
\geometry{includehead, includefoot, 
    left=30mm,right=25mm, top= 10mm, bottom= 10mm}
\setboolean{@twoside}{false} % einseitiges Layout

\usepackage[missing={Help-missing!}]{gitinfo}

\usepackage{fancyhdr}
\pagestyle{fancy}

%%%% floatrow options
%\floatsetup{capposition=top}
\floatsetup[widefigure]{capposition=bottom,floatwidth=\textwidth}
\floatsetup[table]{capposition=top}

%\usepackage[]{caption}
%\captionsetup{labelfont=bf,font={sf,footnotesize},format=plain,labelsep=period}

\lhead{\rightmark}
\rhead{(\the\day.\the\month.\the\year)~~~-\thepage-}
\lfoot{\footnotesize \emph{Git revision\gitVtags:} \parbox[t]{.60\textwidth}{\gitAuthorIsoDate~\gitReferences}}
\cfoot{}
\rfoot{\footnotesize \emph{Git abbrev. hash}: {\ttfamily\gitAbbrevHash}}
\renewcommand{\footrulewidth}{0.4pt}
\renewcommand{\headrulewidth}{0pt}


}{%
    %%%% Textsatz, Layout, Stil
\usepackage[osf]{mathpazo} % auch Palatino, andere math font
\linespread{1.05}         % Palatino needs more leading (space between lines)
%\usepackage[scaled=0.86]{berasans} % serifenlose Schrift
%\usepackage[scaled]{helvet} serifenlose Schrift
\usepackage[defaultsans]{droidsans} % serifenlose Schrift -- DIE HIER wäre schick!
\usepackage[scaled=1.03]{inconsolata}

%Überschriften serifenlos und über den Rand hängend
\usepackage[sf,sl,outermarks,noindentafter,nobottomtitles]{titlesec}

%könnte alles auch mit \bfseries versehen werden nach Geschmack
\titleformat{\section}[hang]{\LARGE\sffamily}{\thetitle}{8pt}{}
\titleformat{\subsection}[hang]{\Large\sffamily}{\thetitle}{8pt}{}
\titleformat{\subsubsection}[hang]{\large\sffamily}{\thetitle}{8pt}{}
\titleformat{\paragraph}[hang]{\bfseries\sffamily}{\thetitle}{8pt}{}
%Etwas aufwendiger für die Kapitelüberschriften:
\titleformat{\chapter}[display]
{\filleft\Huge\sffamily} %Huge ist die Größe für Titeltext und Nummer
{\fontsize{100pt}{90pt}\selectfont\thechapter}
{-2ex} %is vertical space in [display] mode
%Platz vor dem ganzen Krempel
{\vspace{1ex}}
%Platz danach
[\vspace{1ex}]

%%% TOC design
\usepackage[titles]{tocloft}
\setlength{\cftbeforechapskip}{1ex}
\setlength{\cftbeforesecskip}{0.8ex}
\setlength{\cftbeforesubsecskip}{0.8ex}

%\renewcommand{\cftchapfont}{\sffamily\bfseries}
%\renewcommand{\cftchappagefont}{\sffamily}
%\renewcommand{\cftsecfont}{\sffamily}
%\renewcommand{\cftsecpagefont}{\sffamily}
%\renewcommand{\cftsubsecfont}{\sffamily}
%\renewcommand{\cftsubsecpagefont}{\sffamily}

\renewcommand{\cftpnumalign}{l}
\renewcommand{\cftsecdotsep}{\cftnodots}
\renewcommand{\cftsubsecdotsep}{\cftnodots}
\renewcommand{\cftchapleader}{\hspace{2em}}
\renewcommand{\cftsecleader}{\hspace{2em}}
\renewcommand{\cftsubsecleader}{\hspace{2em}}
\renewcommand{\cftchapafterpnum}{\cftparfillskip}
\renewcommand{\cftsecafterpnum}{\cftparfillskip}
\renewcommand{\cftsubsecafterpnum}{\cftparfillskip}

%%%% Grafiken, Abbildungen
\floatsetup{
    capposition=beside,
    capbesideposition={top,outside},
    facing=yes,
    floatwidth=.75\linewidth,
    capbesidewidth=sidefil,
    capbesidesep=quad,
    floatrowsep=quad,
    %framestyle=colorbox,framearound=all,colorframeset=imgbg,frameset={\fboxrule0pt},
    }
\floatsetup[widefigure]{%
    floatwidth=0.95\textwidth,
    %margins=hangoutside,
    %capposition=beside,
    capposition=below,
    %capbesideposition={top,outside},
    %capbesidewidth=\marginparwidth,
    %capbesideframe=yes,
    %capbesidesep=columnsep,
    %floatrowsep=columnsep,
    %heightadjust=nocaption,
    facing=yes,
    }
\floatsetup[widetable]{%
    floatwidth=0.95\textwidth,
    %margins=hangoutside,
    %capposition=beside,
    capposition=above,
    %capbesideposition={top,outside},
    %capbesidewidth=\marginparwidth,
    %capbesideframe=yes,
    %capbesidesep=columnsep,
    %floatrowsep=columnsep,
    %heightadjust=nocaption,
    facing=yes,
    }
\floatsetup[widefloat]{%
    floatwidth=0.95\textwidth,
    %margins=hangoutside,
    %capposition=beside,
    capposition=below,
    %capbesideposition={top,outside},
    %capbesidewidth=\marginparwidth,
    %capbesideframe=yes,
    %capbesidesep=columnsep,
    %floatrowsep=columnsep,
    %heightadjust=nocaption,
    facing=yes,
    }
\usepackage[]{caption}
\DeclareCaptionLabelSeparator{vbar}{ | } % custom; standard z.B. colon, period, ...
\captionsetup{labelfont=bf,font={sf,footnotesize},format=plain,labelsep=vbar}


}

%%%%% Bibliografie
\usepackage[autostyle,english=british,autopunct]{csquotes}
\usepackage[backend=biber,sorting=nyt,
            %style=authoryear-comp,autocite=footnote,
            style=numeric-comp,autocite=plain,
            firstinits=true,uniquename=init,backref=false,
            maxbibnames=25,minbibnames=10,maxcitenames=2,
            url=false,doi=true,isbn=false,eprint=true,
            ]{biblatex}

\addbibresource{zotero-full.bib}

%%% fine-tuning of the appearance
\AtEveryBibitem{\clearfield{month}}
\AtEveryBibitem{\clearfield{day}}
\renewcommand{\labelnamepunct}{\addcolon\addspace}
\DeclareFieldFormat[article]{volume}{\mkbibbold{#1}}
\DeclareCiteCommand{\citepatent}
    [\mkbibfootnote]
    {\usebibmacro{prenote}}
    {patent \printtext[bibhyperref]{\thefield{number}}}
    {\multicitedelim}
    {\usebibmacro{postnote}}
\DeclareCiteCommand{\supercite}
    [\mkbibsuperscript]
    {%
        \usebibmacro{cite:init}%
        \let\multicitedelim=\supercitedelim
        \iffieldundef{prenote}
        {}
        {\BibliographyWarning{Ignoring prenote argument}}%
        \iffieldundef{postnote}
        {}
        {\BibliographyWarning{Ignoring postnote argument}}%
        \bibopenbracket
    }%
    {\usebibmacro{citeindex}%
    \usebibmacro{cite:comp}}
    {}
    {\usebibmacro{cite:dump}\bibclosebracket}

\DeclareCiteCommand{\mycite}
    []
    {\usebibmacro{prenote}}
    {%
        \printnames[][-\value{listtotal}]{author}: %
        \printfield{title}, %
        \iffieldundef{booktitle}
            {\printfield{journaltitle} }
            {\printfield{booktitle} }
        \printfield{volume}.\printfield{number} (\printfield{year}), %
        \printfield{pages}%
    }
    {\multicitedelim}
    {\usebibmacro{postnote}}

%\DeclareCiteCommand{\publistcite}
    %[]
    %{\usebibmacro{prenote}}
    %{%
        %\printnames[][-\value{listtotal}]{author}: %
        %\printfield{title}, %
        %\printfield{journaltitle} \printfield{volume}.\printfield{number} (\printfield{year}), %
        %\printfield{pages}%
        %\printfield{doi}%
    %}
    %{\multicitedelim}
    %{\usebibmacro{postnote}}

\makeatletter
%%%% use maxbibnames instead of maxcitenames in fullcite:
\DeclareCiteCommand{\fullcite}
  {\defcounter{maxnames}{\blx@maxbibnames}%
    \usebibmacro{prenote}}
  {\usedriver
     {\DeclareNameAlias{sortname}{default}}
     {\thefield{entrytype}}}
  {\multicitedelim}
  {\usebibmacro{postnote}}
\makeatother

\let\cite=\autocite  %% supercite als Standard


%%% nachfolgend Umdefinitionen von Unicode-Zeichen, die
%%% in manchen Zotero-BibLaTeX-Einträgen sind:
\DeclareUnicodeCharacter{00A0}{~}  % non-breaking space
\DeclareUnicodeCharacter{202F}{\,} % narrow non-breaking space
\DeclareUnicodeCharacter{2060}{}   % zero-width space
\DeclareUnicodeCharacter{2002}{\:} % en space
\DeclareUnicodeCharacter{2003}{\;} % em space
\DeclareUnicodeCharacter{2007}{ }  % figure space
\DeclareUnicodeCharacter{2009}{\,} % thin space
\DeclareUnicodeCharacter{2010}{-}  % hyphen
\DeclareUnicodeCharacter{2012}{-}  % figure dash
\DeclareUnicodeCharacter{2013}{--} % en dash
\DeclareUnicodeCharacter{2014}{---}% em dash
\DeclareUnicodeCharacter{2015}{-}  % horizontal bar
\DeclareUnicodeCharacter{2212}{-}  % minus

%%%%% Grafiken, Abbildungen
\usepackage[final]{graphicx} % option final to show images also in draft mode
\usepackage{subfig}
\usepackage{wrapfig}
\usepackage{import}
\usepackage{pgf,tikz}
\usetikzlibrary{positioning}
\usetikzlibrary{patterns}
\usetikzlibrary{intersections}
\usetikzlibrary{shadows}
\usetikzlibrary{spy}
\usetikzlibrary{shapes.symbols, shapes.misc, shapes.geometric, shapes.arrows}
\usetikzlibrary{decorations.markings}
\usepgflibrary{decorations.shapes}
\usetikzlibrary{calc}
\tikzset{>=stealth}


%%%% Tabellen,Listen
\usepackage{tabularx,booktabs,multirow}
\usepackage[inline]{enumitem}
\renewcommand{\labelenumi}{(\roman{enumi})}

%%%% Mathe, Zahlen, chem. Formeln
\usepackage{amsmath,amssymb}
\usepackage{eurosym}
\usepackage{dsfont}
\usepackage[abbreviations=true,
            detect-all,
            product-units = brackets,
            list-units=repeat,
            range-units=repeat,
            multi-part-units=brackets,
            per-mode=reciprocal,
            separate-uncertainty =true,
            range-phrase = \text{ to },
            list-final-separator = \text{ and },
           ]{siunitx}
\DeclareSIUnit{\px}{px}
\DeclareSIUnit[number-unit-product = {\thinspace}]{\inch}{inches}
\DeclareSIUnit{\EUR}{\text{\euro}}

\usepackage[version=3]{mhchem}
\usepackage{xfrac}

%%%% Verschiedenes
\usepackage[para,multiple,stable,perpage,symbol*]{footmisc}
%%% footnote without marker
\newcommand\blfootnote[1]{%
  \begingroup
  \renewcommand\thefootnote{}\footnote{#1}%
  \addtocounter{footnote}{-1}%
  \endgroup
}

%%%% Comment-Sections
\usepackage{comment} %% drinnen lassen fuer Lang-Abstract

%%%% Testen (am Ende entfernen)
%\usepackage{showkeys}
%\renewcommand*{\showkeyslabelformat}[1]{%
%\fbox{\normalfont\tiny\ttfamily#1}}
\usepackage[textsize=scriptsize,bordercolor=none,backgroundcolor=YellowGreen,linecolor=YellowGreen]{todonotes}
%\renewcommand{\todo}[1]{\todo{\sffamily #1}}
\newcommand{\dofig}[1]{\todo[backgroundcolor=DarkSeaGreen,linecolor=none]{\sffamily\textbf{DoFigure:}~#1}\xspace}
\newcommand{\dotxt}[1]{\todo[backgroundcolor=Coral,linecolor=none]{\sffamily\textbf{DoText:}~#1}\xspace}
\newcommand{\doref}[1]{\todo[backgroundcolor=Gold,linecolor=Gold]{\sffamily\textbf{DoRef:}~#1}\xspace}
\newcommand{\doalt}[1]{\textcolor{SkyBlue}{#1}\todo[backgroundcolor=SkyBlue,linecolor=SkyBlue]{\sffamily\textbf{Altrn:}~#1}\xspace}

%%%% Hier kommt's auf die Reihenfolge an
\usepackage{varioref}
\usepackage[pdfpagelabels]{hyperref}
%\usepackage{breakurl} % damit URLs korrekt umgebrochen werden
\usepackage[capitalise]{cleveref}

\ifdraft{%
    \hypersetup{%
            pdftitle={},    
            pdfauthor={Anton Haase},
            pdfcreator={},
            pdfborder=0 0 0,
            breaklinks=true,
            bookmarksopen=true,
            bookmarksnumbered=true,
            linkcolor=black,
            urlcolor=SeaGreen,
            citecolor=SteelBlue,
            colorlinks=true}
}{%
    \hypersetup{%
        pdftitle={},    
        pdfauthor={Anton Haase},
        pdfcreator={},
        pdfborder=0 0 0,
        breaklinks=true,
        bookmarksopen=true,
        bookmarksnumbered=true,
        linkcolor=NavyBlue,
        urlcolor=NavyBlue,
        citecolor=NavyBlue,
        colorlinks=false}

 
%%% Kopf- und Fußzeilen
\newlength{\marginWidth}
\setlength\marginWidth{\marginparwidth+\marginparsep}
\newlength{\fulllinewidth}
\setlength\fulllinewidth{\textwidth+\marginWidth}

\usepackage{truncate} %Um zu lange Kapiteltitel abzuschneiden

\footskip=1.6cm
\makeatletter % = mache @ letter 

%Vordefinition mehrfachverwendeter Teile
\def\oddfootSTANDARD{
   \renewcommand{\@oddfoot}{
       \hbox to\textwidth{\vbox{\hbox to\textwidth{
          \hfill
          \strut
          \hspace{1pt}
       }}}
       \hbox to\marginWidth{\vbox{\hbox to\marginWidth{
          \strut %unsichtbares Zeichen
               \large
               \hspace{5pt}               
               \vrule width 1pt height 1cm
            \hspace{8pt}            
            \textsf{\thepage}
            \hfill
       }}}\hss   
   }
}

\def\evenfootSTANDARD{
   \renewcommand{\@evenfoot}{
      \hspace{-\marginWidth}  
         \hbox to\marginWidth{\vbox{\hbox to\marginWidth{
         \large
         \strut %unsichtbares Zeichen
         \hfill
         \textsf{\thepage}
         \hspace{5pt}
         \vrule width 1pt height 1cm
         \hspace{7pt}
      }}}\hss
   }  
}

%Standardstil für die gesamte Dissertation
\newcommand{\ps@thesis}{%
   \renewcommand{\@oddhead}{%
         \hbox to\textwidth{\vbox{\hbox to\textwidth{%
            \textsf
            \hfill
            \rightmark
            \strut
            \hspace{1pt}
      }}}
         \hbox to\marginWidth{\vbox{\hbox to\marginWidth{%
            \strut %unsichtbares Zeichen
            \hspace{5pt}
            \vrule width 1pt
            \hspace{5pt}
            \textsf
            \thesection
            \hfill
         }}}\hss
   }  
   
   \renewcommand{\@evenhead}{%
      \hspace{-\marginWidth} 
         \hbox to\marginWidth{\vbox{\hbox to\marginWidth{%
            \hfill
            \strut %unsichtbares Zeichen
            \textbf{\textsf{Chapter~\thechapter}}
            \hspace{5pt}
            \vrule width 1pt
            \hspace{7pt}
            \strut
         }}}\hss
         
         \hbox to\textwidth{\vbox{\hbox to\textwidth{%
            \strut %unsichtbares Zeichen
         \truncate{.9\textwidth}{\leftmark}
         \hfill
      }}}\hss
   }
   
   \oddfootSTANDARD   
   \evenfootSTANDARD   
}
%Der PLAIN-Style der Chapter- und Sonderseiten muss redefiniert werden.
\renewcommand{\ps@plain}{%
   \let\@oddhead\@empty
   \let\@evenhead\@empty
   \let\@evenfoot\@empty   
   \oddfootSTANDARD
}
%Spezieller Stil für Inhaltsverzeichnis und Literaturverzeichnis (ohne Nummern wie 0.0 oder B.0)
\newcommand{\ps@thesisINTRO}{%
   \renewcommand{\@oddhead}{%
         \hbox to\textwidth{\vbox{\hbox to\textwidth{%
            \textsf
            \hfill
            \sffamily\rightmark
            \strut
            \hspace{1pt}
         }}}\hss
   } 
   
   \renewcommand{\@evenhead}{%
         \hbox to\textwidth{\vbox{\hbox to\textwidth{%
            \strut %unsichtbares Zeichen
            \truncate{.9\textwidth}{\sffamily\leftmark}
            \hfill
         }}}\hss  
   }
   
   \oddfootSTANDARD   
   \evenfootSTANDARD   
}

%Spezieller Stil für Anhänge
\newcommand{\ps@thesisANHANG}{%
   \renewcommand{\@oddhead}{%
         \hbox to\textwidth{\vbox{\hbox to\textwidth{%
            \textsf
            \hfill
            \rightmark
            \strut
            \hspace{1pt}
         }}}
         \hbox to\marginWidth{\vbox{\hbox to\marginWidth{%
            \strut %unsichtbares Zeichen
            \hspace{5pt}
            \vrule width 1pt
            \hspace{5pt}
            \textsf
            \thechapter
            \hfill
         }}}\hss
   }
   
   \renewcommand{\@evenhead}{%
      \hspace{-\marginWidth}  
         \hbox to\marginWidth{\vbox{\hbox to\marginWidth{%
            \hfill
            \strut %unsichtbares Zeichen
            \textbf{\textsf{Appendix~\thechapter}}
            \hspace{5pt}
            \vrule width 1pt
            \hspace{7pt}
            \strut
         }}}\hss
         
         \hbox to\textwidth{\vbox{\hbox to\textwidth{%
            \strut %unsichtbares Zeichen
            \truncate{.9\textwidth}{\leftmark}
            \hfill
         }}}\hss  
   }
   
   \oddfootSTANDARD
   \evenfootSTANDARD
}


\newcommand{\ps@reallyempty}{%
   \let\@oddhead\@empty
   \let\@evenhead\@empty
   \let\@oddfoot\@empty
   \let\@evenfoot\@empty
}

\renewcommand{\chaptermark}[1]{\markboth{\uppercase{\textsf{#1}}}{}}
\renewcommand{\sectionmark}[1]{\markright{\textsf{#1}}}

\makeatother % = mache @ wieder zu nicht-Buchstaben 
\pagestyle{thesis}


%Problem mit den Seitenzahlen und Headern auf leeren Seiten nach Kapiteln:
\let\origdoublepage\cleardoublepage
\newcommand{\clearemptydoublepage}{%
  \clearpage
  {\pagestyle{empty}\origdoublepage}%
}
\let\cleardoublepage\clearemptydoublepage

}

%%% glossaries for Abbreviations, Glossary
\usepackage[nonumberlist, nomain, nogroupskip]{glossaries}
%\usepackage{acronym}
\newglossary[alg]{acronym}{acr}{acn}{\acronymname}
\newglossary[slg]{symbols}{sls}{slo}{\glssymbolsgroupname}
\makeglossaries


%%%%%%%%%%%%%%%%%%%%%%
%%%%% eigene Kommandos
\usepackage{overpic} %for draft text overlays
\usepackage{rotating}
\newcommand{\draftImage}[2]{%
    \begin{overpic}[#1]{#2}
        \put(0,0){\includegraphics[#1]{#2}}
        \put(2,2){\LARGE\color{CornflowerBlue}{\ttfamily\begin{rotate}{45}[Draft]\end{rotate}}}
    \end{overpic}
}
\newcommand{\CD}{\ensuremath{C\mskip-3mu D}\xspace}
\renewcommand{\phi}{\ensuremath{\varphi}\xspace}
\newcommand{\vidx}[2]{\ensuremath{#1_\text{#2}}\xspace}
\newcommand{\eph}{\vidx{E}{ph}}
\newcommand{\ethr}{\vidx{E}{thresh}}
\newcommand{\ali}{\vidx{\alpha}{i}}
\newcommand{\alc}{\vidx{\alpha}{c}}
\newcommand{\alf}{\vidx{\alpha}{f}}
\newcommand{\thf}{\vidx{\theta}{f}}
\newcommand{\dens}{\ensuremath{\varrho}\xspace}
\newcommand{\edens}{\ensuremath{\vidx{\varrho}{e}}\xspace}
\newcommand{\pdepth}{\ensuremath{\Lambda}\xspace}
\newcommand{\ls}{\vidx{L}{s}}
\newcommand{\lpx}{\vidx{L}{px}}
\newcommand{\q}[1]{\vidx{q}{#1}}
\newcommand{\qval}[1]{\ensuremath{\SI[per-mode=reciprocal]{#1}{\per\nm}}\xspace}
\newcommand{\ev}[1]{\ensuremath{\SI{#1}{\eV}}\xspace}
\newcommand{\kev}[1]{\ensuremath{\SI{#1}{\keV}}\xspace}
\newcommand{\qe}{\ensuremath{\mathit{QE}}\xspace}
\newcommand{\nm}[1]{\ensuremath{\SI{#1}{\nm}}\xspace}
\newcommand{\mm}[1]{\ensuremath{\SI{#1}{\mm}}\xspace}
\newcommand{\m}[1]{\ensuremath{\SI{#1}{\m}}\xspace}
\newcommand{\mrad}[1]{\ensuremath{\SI{#1}{\milli\radian}}\xspace}
\newcommand{\mbar}[1]{\ensuremath{\SI{#1}{\milli\bar}}\xspace}

\newcommand{\pvp}{PS-\textit{b}-P2VP\xspace}
\newcommand{\pil}{PILATUS~1M\xspace}
\newcommand{\vpil}{in-vacuum \pil}

\newcommand{\ivec}[2]{\ensuremath{\vidx{\vec{#1}}{#2}}\xspace}
\newcommand{\ivecabs}[2]{\ensuremath{|\ivec{#1}{#2}|}\xspace}
\newcommand{\n}{\ensuremath{\hat{n}}\xspace}
\newcommand{\ft}[1]{\ensuremath{\mathcal{F}(#1)}\xspace}
\newcommand{\dft}[1]{\ensuremath{\mathcal{F}_{\text{DFT}}(#1)}\xspace}
%\newcommand{\psd}[1]{\ensuremath{\mathcal{PSD}(#1)}\xspace}
\newcommand{\psd}[1]{\ensuremath{\mathrm{PSD}(#1)}\xspace}
\newcommand{\hkl}[1]{\ensuremath{\left(#1\right)}\xspace}
\newcommand{\imagu}{\ensuremath{i}\xspace}
\renewcommand{\Re}{\operatorname{\mathfrak{R}}}
\renewcommand{\Im}{\operatorname{\mathfrak{I}}}
\newcommand{\rarr}{\ensuremath{\curvearrowright}\xspace}

\definecolor{data_color}{HTML}{A60628} 
\definecolor{fit_color}{HTML}{348ABD} 
