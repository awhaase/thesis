%%%%% Glossary entries
\newcommand{\newsymentry}[4]{\newglossaryentry{#1}{name=\ensuremath{#2}, description={#3}, sort=#4, type=symbols}}

%% Acronyms
\newacronym{dwba}{DWBA}{distorted-wave Born approximation}
\newacronym{euv}{EUV}{exreme ultraviolet}
\newacronym{xrr}{XRR}{X-ray reflectivity}
\newacronym{eV}{eV}{electron volt}
\newacronym{xrf}{XRF}{X-ray fluorescence}
\newacronym{gixrf}{GIXRF}{grazing incidence X-ray fluorescence}

%%% Symbols
\newsymentry{dcs}{\ensuremath{\frac{d\sigma}{d\Omega}}}{differential scattering cross section}{sigmaOmega}
\newsymentry{lambda}{\ensuremath{\lambda}}{wavelength}{lambda}
\newsymentry{q}{\ensuremath{\vec{q}}}{\mbox{scattering vector / reciprocal space vector}, ${\vec{q} = \left(\q{x},\q{y},\q{z}\right)^{T}}$}{q}
\newsymentry{k}{\ensuremath{\vec{k}}}{wave vector; the wave number is \mbox{$|\vec{k}| = \gls{k_0} = \sfrac{2\pi}{\gls{lambda}}$}}{k}
\newsymentry{omega}{\ensuremath{\omega}}{frequency}{omega}
\newsymentry{k_0}{\ensuremath{k_0}}{modulus of the wave vector in vacuum (wave number) \mbox{$k_0 = \sfrac{\gls{omega}}{\gls{c}} = \sfrac{2\pi}{\gls{lambda}}$}}{k_0}
\newsymentry{alpha_i}{\ensuremath{\alpha_i}}{angle of incidence defined from the surface normal}{alpha_i}
\newsymentry{alpha_f}{\ensuremath{\alpha_f}}{scattering angle defined from the surface normal}{alpha_f}
\newsymentry{theta_f}{\ensuremath{\theta_f}}{azimutal scattering angle (out-of-plane scattering angle)}{theta_f}
\newsymentry{n}{\ensuremath{n}}{complex index of refraction, $n = \delta + i \beta$}{n}
\newsymentry{epsilon_0}{\ensuremath{\epsilon_0}}{vacuum permittivity or electric constant}{epsilon_0}
\newsymentry{mu_0}{\ensuremath{\mu_0}}{vacuum permeability or magnetic constant}{mu_0}
\newsymentry{c}{\ensuremath{c}}{speed of light in vacuum \mbox{$c = \sfrac{1}{\sqrt{\gls{epsilon_0} \gls{mu_0}}}$}}{c}
\newsymentry{h}{\ensuremath{h}}{Planck's constant, $h=4.135667662(25)\times10^{-15}$ eV$\,$s}{h}
\newsymentry{el_dens}{\ensuremath{\rho_e(\vec{r})}}{electron density at position $\vec{r}$}{rho}
\newsymentry{r_e}{\ensuremath{r_e}}{classical electron radius \mbox{$r_e = \sfrac{e^2}{4 \pi \epsilon_0 m c^2} = 2.82 \times 10^{-5} \AA$}}{r_0}
%% Glossary
%\newglossaryentry{pilatus}{name=PILATUS, description={modular two-dimensional hybrid pixel X-ray detector}}
%\newglossaryentry{ewaldsphere}{name=Ewald sphere, description={construct to visualise the occurrence of scattering spots, the radius is $\sfrac{2\pi}\lambda$}}
