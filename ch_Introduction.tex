\glsresetall
\chapter{Introduction} \label{ch:Intro}
In 1959, Jack S.~Kilby made an invention that should revolutionize the world in the years to come. His development of the first integrated circuit was the realization of a logical element known as \emph{flip-flop}, capable of storing a single bit, by implementing a layout that could host all required circuits on a single semiconductor wafer piece  \cite{kilby_invention_1976}. His achievement paved the way for the miniaturization of electronic circuits that enabled the technological advancements we experienced over the past 57 years and was awarded as part of the Nobel prize in physics in 2000 \cite{noauthor_press_nodate}. Only two years after the original invention, Robert N.~Noyce submitted a patent on the fabrication of integrated circuits in monolithic single crystals using photo lithography to create the necessary artificial structure \cite{noyce_semiconductor_1961}. This technique of using light to transfer a pattern from a photomask onto a semiconductor wafer has prevailed over the course of the development and is still the primary method on the fabrication of computer chips today \cite{mack_fundamental_2008}. As the technology improved over time it roughly followed \emph{Moore's law} of doubling the transistor count on a unit area of the wafer every two years \cite{moore_cramming_1998}. In consequence, the structured feature sizes on the wafers shrank to down to accommodate this large amount of circuits on a single chip. Today, structure sizes of only few tenth of nanometer have been reached \cite{international_roadmap_committee_international_2015}. With this strong decrease in size, technological requirements on the lithography systems used to fabricate those chips in mass production grew.

A basic principle of optical resolution known as the \emph{Rayleigh criterion} states, that the minimum structure size achievable with an optical system is roughly proportional to the wavelength used \cite{lord_rayleigh_xxxi._1879}. Consequently, while the first lithography systems used in the semiconductor industry had wavelength in the visible spectrum, wavelengths had to be reduced down to the \gls{duv} wavelength of $\nm{193}$ reached nowadays to keep pace with Moore's law. However, with feature sizes of only a few tenths of nanometer now necessary, a significant further reduction of the wavelength can not be avoided. The next-generation optical lithography uses wavelengths in the \gls{euv} spectral range of \nm{13.5}. This radiation is strongly absorbed by all materials, including air, challenging the design of the optical lithography systems by effectively ruling out any optical design based on transmission lenses for focusing and imaging of the structures.

With the semiconductor industry at the verge of a major technological change, the topic of reflective optical elements for \gls{euv} radiation has gained a large amount of attention and experienced extensive research efforts \cite{bakshi_euv_2009}. Back in 1972, Eberhart Spiller had proposed a new design for efficient mirror systems working at incidence angles near the surface normal. The idea was based on fabricating artificial layer systems reflecting portions of the incoming radiation at each interface that would interfere constructively at acceptable absorption levels overcoming the extremely low reflection otherwise seen from single surfaces \cite{spiller_low-loss_1972}. The result are bandwidth limited multilayer Bragg reflectors which fulfill the Bragg condition for constructive interference and thus require specific design for each wavelength and angle of incidence. At angles close to the surface normal, layers of thicknesses in the order of half the design wavelength are necessary, which requires fabrication methods capable of precisely depositing layers of only several nanometers. Since the original proposal, multilayer systems have been realized using evaporation and sputtering techniques and demonstrated to increase reflection \cite{spiller_reflective_1976, underwood_layered_1981}. As the technology developed and more advanced sputtering techniques became available to fabricate at the necessary precision \cite{stearns_fabrication_1991}, the first important applications of focusing multilayer mirrors were space probes used for the observation of the sun in the \gls{euv} spectrum \cite{chauvineau_description_1992, clette_eit:_1995, spiller_soft_1994}.

Theoretical models and calculations of candidate systems for large reflectivity close to normal incidence at a wavelength of \nm{13.5}, show peak values of approximately $\SI{72}{\percent}$ by using multilayer systems based on molybdenum (Mo) and silicon (Si). State of the art systems fabricated today reach values closely above $\SI{70}{\percent}$ \cite{barbee_molybdenum-silicon_1985,stearns_fabrication_1991,bajt_improved_2002,braun_grenzflachen-optimierte_2003}, which is still a few percentage points below the theoretical limit. This is of particular concern for the usage in \gls{euv} lithography systems, where $6$ to $9$ reflections are required to image a structure onto a wafer \cite{kaiser_euvl_2008, wagner_euv_2010}. The small difference to the theoretical reflection limit thus has a large impact on the total radiant power at the wafer level and is a very crucial point in the development of the next-generation lithography using \gls{euv} radiation. 

While the semiconductor industry undoubtedly is a very strong driving force in the development of \gls{euv} multilayer optics for \nm{13.5} wavelength,  multilayer mirrors for other spectral ranges suffer from the same issue. A system of particular interest for this work is a mirror designed to reflect radiation in the range of the so-called \emph{water window} between $2.3$ nm and $4.4$ nm. The water window is of special interest, because radiation in this spectral range shows low absorption in water, while it is absorbed by many elements naturally occurring in organic molecules such as proteins \cite{kirz_soft_1995}. This allows the study of biological systems in their native environment (water), where many proteins are biologically active. With the ability to produce radiation at those wavelengths at \gls{fel} sources \cite{ackermann_operation_2007, schreiber_first_2011}, more applications with strong and coherent pulses come in reach. High resolution imaging of protein samples, in addition to requiring the short wavelength, needs sufficient reflected radiation intensity and, more generally, optical elements capable of focusing and magnification. This can be achieved with high reflectance multilayer mirrors \cite{hertz_normal-incidence_1999,legall_compact_2012}. A candidate system relevant in this range is made from chromium (Cr) and scandium (Sc) applying the very same principle as introduced above, however, having much thinner layer thicknesses due to the shorter wavelength. While the theoretical reflectance limit at \nm{3.1} wavelength is calculated to reach values above $\SI{50}{\percent}$ \cite{schafers_cr/sc_1998}, state of the art samples only show reflectivities below \SI{20}{\percent} \cite{eriksson_14.5_2003, yulin_high-performance_2004}, less than half of the theoretically possible values.

The main reasons for radiation loss beyond unavoidable absorption inside the materials of both the multilayer systems are imperfections at the interfaces, such as compound formation, intermixing and roughness. As a result, the perfect multilayer system is distorted, since the interfaces are not chemically abrupt anymore. Thus, intermixing leads to a diminished optical contrast and consequently to lower reflectance at the respective interface \cite{nakajima_interdiffusion_1988}. Interdiffusion is a known problem for multilayer optics, and measures to counteract this effect are the introduction of barrier layers hindering the formation of intermixing layers in some of the systems \cite{braun_grenzflachen-optimierte_2003,braun_mo/si_2002}. In the case of roughness, the result of reduced optical contrast at the interfaces is the same on average for the impinging wavefield, however, in this case with the addition of diffuse scattering outside the specular beam direction \cite{sinha_x-ray_1994}, which is not present in the case of pure intermixing.

To minimize interface distortions and to ultimately increase the reflectivity of the respective systems, the research and industry groups concerned with fabricating multilayer mirrors require detailed information on the actual structural properties and the interface morphology of their samples. The characterization of those multilayer systems is thus a cornerstone in the effort for improvement and the fundamental understanding of the effects involved. As the German \gls{nmi}, the \gls{ptb} is dedicated to precise measurements related to all fields of physics and technology providing metrology as its core mission. In fact, the international metrology organization, the Bureau International des Poids et Mesures, defines\footnote{Source: \url{http://www.bipm.org/en/worldwide-metrology/}} metrology as \emph{``the science of measurement, embracing both experimental and theoretical determinations at any level of uncertainty in any field of science and technology.''}. In this spirit, this thesis seeks to provide metrology for the important field of multilayer optics introduced above.

There are several characterization techniques that exist and have been applied to assess and quantify roughness and intermixing of materials at the interfaces of multilayer mirrors in the past. Some widely used example is \gls{tem}, which establishes a microscopic approach to the problem of assessing the interface morphology with resolution at the nanoscale \cite{stearns_thermally_1990, bajt_investigation_2001}. By imaging the layer stack, interface imperfections are directly visible. In combination with \gls{hreels}, element specific interface profiles can be deducted giving insight into the intermixing behavior of two (or more) materials at the interfaces \cite{egerton_electron_2011, prasciolu_thermal_2014}. A large downside of both methods, however, is the intrinsically local area of the image and thus the characterization of only very small local portions of the entire sample. Apart from that, the stack needs to be cut open to apply these techniques and thus leads to a destruction of the sample.

Another very popular method often applied before and after deposition of a multilayer stack is \gls{afm} \cite{binnig_atomic_1986}. It is a scanning technique with nanometer resolution allowing to determine the morphology of a surface and thus investigate their roughness. However, it faces the same locality obstacle as \gls{tem} and can only operate on exposed areas. Thus, the morphology of buried structures remains hidden from this method. Nevertheless, it is applied to determine the initial substrate roughness and the condition of the final top surface as an important prerequisite for high-quality multilayer mirror fabrication \cite{louis_progress_2000, bajt_investigation_2001}.

Apart from the direct and local scanning techniques, indirect ensemble methods based on the elastic scattering of radiation are accurate and extensively used in multilayer characterization. Those include \gls{xrr} and \gls{euv} reflectivity with \gls{reuv} as a variation of the latter. They are employed as standard approach in multilayer mirror fabrication \cite{lim_fabrication_2001, bajt_investigation_2001, braun_mo/si_2002}. The major advantaged is, that they are destruction-free, contact less and deliver information on the buried structure as well as on the top surface condition. Furthermore, statistical information across a large area depending on the beam footprint of the impinging radiation is obtained in contrast to the aforementioned local methods. However, it is no longer possible to directly gain information on the multilayer stack as theoretical models parameterizing its structural properties are required to calculate the measured reflectivity curves. Reconstruction of the model parameters by fitting calculations to the experimental data raises the question of uniqueness and accuracy of the solution found. Even the applicability of the model and its limitations itself are of importance in these considerations. Several researches have shown, that the combination of \gls{euv} and \gls{xrr} can lead to significant improvements in the accuracy compared to standalone measurements with each technique individually \cite{yakunin_combined_2014}. \Gls{xrf} is another technique based on detecting fluorescence radiation of the materials inside the multilayer stack, that has been applied to add further complementary information to assist in the solution of this problem \cite{kortright_standing_1987, kawamura_interface_1994, ghose_x-ray_2001}.

While structural information on the layer stack can be obtained through reconstruction of a model by conducting these experiments, only limited information is gained on the roughness of the interfaces which are difficult to distinguish from intermixing. However, as only roughness causes diffuse scattering, the analysis of this off-specular intensity upon irradiation of a multilayer stack is a natural tool for characterization of the interface morphology. A lot of theoretical and experimental work has been conducted in relation to the study of diffuse scattering from multilayer samples, mostly in grazing incidence geometries using x-rays \cite{mikulik_x-ray_1997, sinha_x-ray_1994, de_boer_x-ray_1995, de_boer_x-ray_1996}, but also in the optical and \gls{euv} regime \cite{amra_light_1993, amra_light_1994, elson_light_1980, elson_relationship_1983, schroder_angle-resolved_2011, schroder_spectral_2014}, to deduct the desired information on the interface roughness.

This thesis is dedicated to the accurate and complete characterization of the structural properties and the interface morphology of multilayer mirrors based on the combination of several of the aforementioned methods. In the \gls{ptb} laboratories at the \gls{mls} and the \gls{bessy}, radiation in the spectral range from the terahertz up to the x-ray regime is available at several dedicated beamlines. Experiments implementing \gls{euv} and \gls{reuv} reflectivity, \gls{euv} diffuse scattering as well as \gls{xrf} have been conducted at the respective specialized end stations. External \gls{xrr} data was incorporated into the analysis to further strengthen the findings made. Experimental uncertainties inevitably associated with any measurement and model uncertainties are investigated with respect to each method employed. Based on different theoretical optimization algorithms, confidence intervals for each reconstructed parameter of the underlying models are determined. As the layer thicknesses enter the sub-nanometer regime for Cr/Sc multilayer mirrors designed for the water window regime, a new model is proposed and evaluated. As a result, all of the methods are assessed with respect to their applicability for the structural investigation of such samples. Based on the reconstruction, the interface morphology is studied using \gls{euv} diffuse scattering analyzed based on the theoretical framework provided by the \gls{dwba} \cite{holy_nonspecular_1994, holy_x-ray_1993}. Again, the resulting parameter values are evaluated with respect to their validity by determining their confidence intervals based on the available data.

This work is structured based on the classification of the characterization methods rather than the samples under investigation. Chapter~\ref{ch_theo} introduces the fundamental theoretical concepts underlying the interaction of multilayer systems with \gls{euv} and x-ray radiation. The theoretical basis of the analytic experiments (\gls{euv} reflectivity, \gls{reuv}, \gls{xrr}, \gls{xrf} and \gls{euv} diffuse scattering) conduced in this thesis to characterize the various samples is given. In chapter~\ref{ch_exp}, the different experimental setups at the two storage rings \gls{mls} and \gls{bessy} employed in obtaining the data analyzed here are presented. In this work samples fitting in two major categories of multilayer mirrors for two different spectral ranges were investigated. Their fabrication was conduced using a sputtering technique, which is briefly reviewed. Furthermore, the extensive software that was developed over the course of this thesis is summarized there. The structural reconstruction of the Mo/Si and Cr/Sc multilayer mirrors based on the combination of the different experiments is presented in chapter~\ref{ch_spec}. Here, the validity of the models and the accuracy of the reconstructed parameters with their confidence intervals is discussed in depth. Finally, chapter~\ref{ch_diff} addresses the evaluation of the interface morphology of the very same samples based on the \gls{euv} diffuse scattering measurements and the models reconstructed in the previous chapter. The summary and conclusion of this thesis can be found in the last chapter~\ref{ch_summary}. Most of the work reported on here has been published in peer-reviewed journals \cite{haase_role_2014, haase_multiparameter_2016, haase_interface_2017}. At the end of each major section within the respective chapters, the corresponding publication related to the results presented is given in detail.
% 
%  
% 
% \paragraph{Notes}
% 
% Standard characterization methods such as EUV reflectance and X-ray reflectance 
% (XRR) with simple binary layer models have proven useful for the 
% characterization of similar multilayer systems, e.g.~Mo/Si mirrors designed for 
% $13.5$ nm wavelength \cite{lim_fabrication_2001, bajt_investigation_2001, braun_mo/si_2002}.
% 
% diffuse scattering bla bla
% 
% Chapter~\ref{ch_theo} 
% 
% 
% 
% \paragraph{Intro Kiessig-like peaks}
% Multilayer systems have been of great interest over the past decades. The first applications of multilayers serving as mirrors for soft X-rays were optical components for space probes. The main driving force today is the shift in direction towards the EUV spectral range at 13.5 nm wavelength in optical lithography for the semiconductor industry. Lenses in classic lithography systems are replaced by multilayer mirrors. High  reflectivities are achieved by utilizing the constructive interference of the reflected light at each interface when fulfilling the Bragg condition. State of the art Mo/Si multilayer mirrors reach reflectivities of up to 70\% \cite{braun_mo/si_2002, feigl_euv_2006} in the case of near-normal incidence EUV radiation. This value is still well below the theoretical limit of approx.~75\% for an ideal multilayer. An important reason for the loss of reflectivity is interface imperfections such as roughness and interdiffusion causing diffuse scattering. The analysis of the off-
% specular scattering 
% thus serves as a 
% natural tool for the characterization of interfacial roughness. 
% 
% At the Physikalisch-Technische Bundesanstalt (PTB), angle and energy resolved scatterometric measurements have been performed to analyze the off-specular scattering using EUV radiation. The tunability of synchrotron radiation in conjunction with angular resolution allows obtaining two-dimensional intensity maps close to the relevant multilayer resonance for near-normal incidence geometries. The diffuse scattering from interface roughness contains information on its morphology, such as lateral and vertical correlations, its jaggedness, and mean amplitude. A rigorous analysis of the reciprocal space represented through the scattering pattern, thus, provides access to the interface morphology.
% 
% Scatterometry poses an inverse problem of gaining information about the properties of the interfaces. A theoretical model of the diffuse scattering is required to yield a reconstruction of the actual sample and deduct the power spectral density (PSD) of roughness. The topic of experimental and theoretical analysis towards the characterization of roughness involving optical wavelengths has been largely studied by others and published in the optical community \cite{amra_light_1993, amra_light_1994, elson_light_1980, elson_relationship_1983, schroder_angle-resolved_2011, schroder_spectral_2014}. We take a different approach involving the analysis of diffuse EUV scattering employing numerical simulations of the expected scattering distribution based on the distorted-wave Born approximation (DWBA)~\cite{holy_nonspecular_1994, holy_x-ray_1993}.
% 
% We will show that a rigorous, dynamic calculation of the EUV radiation interacting with the multilayer is required to obtain the power spectral densities. The influence of multiple reflections at the layer boundaries cannot be neglected in this analysis. The simulations are compared to measured data obtained for high-reflectance Mo/Si multilayers. The influence of the measurement geometry of the diffuse scattering is discussed in detail.
% 
% \paragraph{Intro Mo/Si}
% Magnetron sputtered EUV multilayer thin-film systems serve as well-established optical elements for near-normal incidence mirrors in the EUV spectral range \cite{martinez-galarce_high_2000,barbee_jr._multi-spectral_1991,toyoda_soft-x-ray_2000,finkenthal_near_1990}. Since the reflectance of single metallic surfaces is negligible for these wavelengths, those systems, instead, provide a periodic arrangement of interfaces between typically two materials with a significantly different index of refraction. Thereby, an artificial one-dimensional Bragg crystal is formed, leading to constructive interference for a designated angle of incidence and wavelength range \cite{spiller_low-loss_1972}. A system of great relevance nowadays is the multilayer of Mo/Si alternating layers used to design near-normal incidence mirrors for the EUV wavelength at $13.5$ nm. The ability to design high-reflectance mirrors for this wavelength has a large impact on the future of the semiconductor industry, where integrated circuits are fabricated using optical lithography on the nano scale. With regularly shrinking feature sizes on silicon wafers, a shorter wavelength than the established 193 nm DUV lithography is required to reach the designated level of miniaturization. The next step is EUV lithography at $13.5$ nm wavelength with radiation produced using laser plasma sources. Because of the limited radiant power of the sources, the margins for radiation loss inside optical elements are minimal. Mo/Si multilayers with additional diffusion barriers at selected interfaces have demonstrated a yield of reflectances above $70\%$ close to the theoretical threshold at near-normal incidence \cite{barbee_molybdenum-silicon_1985,stearns_fabrication_1991,bajt_improved_2002,braun_grenzflachen-optimierte_2003}. The period thickness required for high near-normal incidence reflectance at 13.5 nm wavelength is at approx.~ $D=7$ nm, in order to achieve constructive interference with $N=50$ multilayer periods. The main reasons for radiation loss beyond unavoidable absorption inside the materials of the multilayer are imperfections at the interfaces, such as interdiffusion and roughness. In both cases the perfect multilayer system is distorted, since the interfaces are not chemically abrupt anymore. Thus, interdiffusion leads to a diminished optical contrast and consequently to lower reflectance at the respective interface \cite{nakajima_interdiffusion_1988}. Interdiffusion is a known problem for multilayer optics, and measures to counteract this effect are the introduction of barrier layers hindering the formation of interdiffusion layers \cite{braun_grenzflachen-optimierte_2003,braun_mo/si_2002}. In the case of roughness, the result of reduced optical contrast at the interfaces is the same on average due to the finite beam size, however, in this case with the addition of diffuse scattering outside the specular beam direction \cite{sinha_x-ray_1994}, which is not present in the case of pure interdiffusion.
% 
% Analyzing the diffuse scattering pattern provides valuable information on the interface morphology in terms of the power spectral density (PSD) of roughness and thus the cause of roughness-induced reflectance loss. We investigate several samples of Mo/Si multilayer systems with C interdiffusion barriers at the Mo on Si interfaces. It has been shown that optimal reflectance can be achieved with approximately $40 \%$ Mo layer thickness with respect to (w.r.t.) the total period thickness \cite{bajt_investigation_2001,braun_mo/si_2002}. The samples under investigation here were fabricated with increasing relative Mo thickness while keeping the nominal period thickness $D\approx 7$ nm constant. It has been observed that with increasing Mo layer thickness, crystallites start forming at a certain threshold during the sample preparation inside the Mo layer \cite{bajt_investigation_2001}. This has an impact on the interface morphology and potentially increases the roughness and thus the loss of specularly reflected radiation. In this study, we investigate two sets of samples with increasing Mo layer thickness as described above. In the first set, the magnetron sputtered layers were deposited one after another for each sample. In the second set, during deposition, an additional polishing process was used after sputtering each layer. To investigate the interface morphology and to observe the amorphous-to-crystalline transition in each set, we measure and analyze the diffuse scattering pattern. The theoretical diffuse scattering maps expected from a certain multilayer model are calculated based on the distorted-wave Born approximation (DWBA) to deduct the PSD by reconstruction \cite{holy_nonspecular_1994,holy_x-ray_1993,haase_role_2014}. To obtain the actual layer thicknesses in the samples, we applied EUV reflectivity and X-ray reflectivity (XRR) experiments and reconstructed these parameters by modeling and by combined analysis of the measured data. The analysis was carried out by using a Markov-chain Monte Carlo method (MCMC) \cite{foreman-mackey_emcee:_2013} to deduct reliable parameters including confidence intervals. Based on these models, the diffuse scattering was calculated and the parameters of the PSD and the vertical roughness correlation were reconstructed again using the MCMC method.
% 
% \paragraph{Intro Cr/Sc}
% The wavelength range of the so-called ``water window'' between $2.3$ nm and 
% $4.4$ nm is of special interest, because radiation in this spectral range shows 
% low absorption in water, while it is absorbed by many elements naturally 
% occurring in organic molecules such as proteins \cite{kirz_soft_1995}. 
% This allows the study of biological systems in their native environment 
% (water), where many proteins are biologically active. In addition to the short 
% wavelength required to achieve high resolution imaging of such samples, one 
% also needs sufficient intensity, which can be achieved with high reflectance 
% optical elements \cite{hertz_normal-incidence_1999,legall_compact_2012}.
% 
% The strong absorption of soft X-ray radiation in most materials poses a 
% challenge in the fabrication of such optics. Refractive optical elements are 
% not available due to the high absorption in solids. The same holds for 
% reflective optical elements close to normal incidence, where reflectivities 
% from a single surface are well below $10^{-4}$ for all materials \cite{henke_x-ray_1993}. 
% A candidate system for building highly-reflective mirrors for short wavelengths 
% is a layered structure with alternating materials of significantly different 
% refractive indices \cite{spiller_low-loss_1972}. Such multiple repeated bilayer 
% systems constitute an artificial one-dimensional Bragg crystal. Their layer 
% layout, more specifically the total layer thickness $D$ of a single layer 
% period, is intrinsically related to the desired peak reflectance wavelength and 
% incidence angle. Such systems are well established as mirrors for an EUV 
% wavelength of $13.5$ nm, where they reflect more than $60\%$ of the radiation 
% close to normal incidence with a choice of Mo and Si as layer materials 
% \cite{barbee_molybdenum-silicon_1985,stearns_fabrication_1991}. By applying additional interface 
% shaping techniques and adding barrier layers to prevent interdiffusion, 
% reflectivities of above $70\%$, close to the theoretical limit, are achievable 
% \cite{bajt_improved_2002,braun_grenzflachen-optimierte_2003}.
% 
% Theoretical calculations show that constructing multilayer mirrors in the water 
% window spectral range for normal incidence allows peak reflectivities above 
% $50\%$ \cite{schafers_cr/sc_1998}. A typical choice of materials for these bilayer 
% systems is Cr and Sc for wavelengths above $3.1$ nm \cite{salashchenko_short-period_1997, 
% schafers_cr/sc_1998}. The proximity to the Sc L edge causes the required significant 
% difference in the refractive index due to anomalous dispersion while 
% maintaining relatively low absorption. In order to function as a 
% one-dimensional Bragg crystal, those layer systems demand a high quality of the 
% layer interfaces. Chemically abrupt and smooth interfaces are required to reach 
% high reflectivities and to minimize loss processes such as diffuse scattering 
% or contrast reduction due to interdiffusion. This requirement becomes even more 
% stringent when moving towards shorter wavelengths due to the necessary 
% reduction in layer thickness for fulfilling the Bragg condition.
% The relative influence of interface morphology and interdiffusion as loss 
% mechanisms for peak reflectance rises in importance compared to established 
% Mo/Si multilayer systems with significantly larger layer thicknesses. The 
% measured peak reflectance of state-of-the-art Cr/Sc multilayer systems designed 
% for the above specifications scores at reflectivities below $20\%$, less than 
% half of the theoretically possible value \cite{eriksson_14.5_2003, 
% yulin_high-performance_2004}.
% 
% Roughness causes diffuse scattering out of the specular beam direction 
% \cite{sinha_x-ray_1994}. Interdiffusion, on the other hand, 
% reduces the optical contrast, i.e.~the local difference in the refractive 
% index, thereby reducing the reflectance at each interface 
% \cite{nakajima_interdiffusion_1988}. In order to gain a deeper understanding of the 
% interface morphology, a characterization of the individual contributions of 
% interface diffusion and roughness is required. Both lead to a damping of the 
% peak reflectance \cite{croce_p._etude_1976}. The inspection of diffusely scattered 
% light is a natural tool for the investigation of the roughness at the 
% interfaces. At-wavelength in-plane diffuse scattering contains information on 
% the interface morphology. An important advantage of this analysis as compared 
% to established methods for interface characterization of thin films such as 
% grazing-incidence small-angle X-ray scattering (GISAXS) \cite{levine_grazing-incidence_1989} is 
% that the angle of incidence is close to the surface normal. This allows the 
% investigation of the multilayer stack locally even for strongly curved 
% surfaces, e.g.~in the case of focusing optics.
% 
% 
% Standard characterization methods such as EUV reflectance and X-ray reflectance 
% (XRR) with simple binary layer models have proven useful for the 
% characterization of similar multilayer systems, e.g.~Mo/Si mirrors designed for 
% $13.5$ nm wavelength \cite{lim_fabrication_2001, bajt_investigation_2001, braun_mo/si_2002}. However, these systems typically have 
% thicknesses of $3$ to $4$ nm for the individual Mo and Si layers. With the 
% efforts of reducing the peak reflectance wavelength, those methods fail to 
% yield consistent information in the framework of simple models that describe the measured 
% reflectivities. This has already been observed in the case of La/B multilayer 
% mirrors designed for peak reflectivities at $6.7$ nm wavelength 
% \cite{yakunin_combined_2014}.
% 
% The reason for this might be an increase in disturbances at the interfaces, 
% which potentially break the symmetry condition. This needs to be taken into 
% account explicitly in the model and leaves the simple binary approach with 
% Nevot-Croce damping factors as an insufficient description of the physical 
% situation. However, the increased number of parameters required to describe 
% such a realistic model also requires more data (information) from analytical 
% measurements. We thus apply a set of different experimental methods to obtain a 
% consistent reconstruction of the multilayer structure with a non-destructive 
% approach. We demonstrate that in the case of layer systems in the subnanometer 
% region, a combined analysis of these experiments is required. We describe the 
% layer system with graded interface profiles to account for the intermixing of 
% the two materials. The validation of the derived model is conducted by applying 
% a Markov-chain Monte Carlo sampler.