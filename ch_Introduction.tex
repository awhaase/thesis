\glsresetall
\chapter{Introduction} \label{ch:Intro}
In 1959, Jack S.~Kilby made an invention that should revolutionize the world in the years to come. His development of the first integrated circuit was the realization of a logical element known as \emph{flip-flop}, capable of storing a single bit, by implementing a layout that could host all required circuits on a single semiconductor wafer piece  \cite{kilby_invention_1976}. His achievement paved the way for the miniaturization of electronic circuits that enabled the technological advancements we experienced over the past 57 years and was awarded as part of the Nobel prize in physics in 2000 \cite{noauthor_press_nodate}. Only two years after the original invention, Robert N.~Noyce submitted a patent on the fabrication of integrated circuits in monolithic single crystals using photo lithography to create the necessary artificial structure \cite{noyce_semiconductor_1961}. This technique of using light to transfer a pattern from a photomask onto a semiconductor wafer has prevailed over the course of the development and is still the primary method on the fabrication of computer chips today \cite{mack_fundamental_2008}. As the technology improved over time it roughly followed \emph{Moore's law} of doubling the transistor count on a unit area of the wafer every two years \cite{moore_cramming_1998}. In consequence, the structured feature sizes on the wafers shrank to down to accommodate this large amount of circuits on a single chip. Today, structure sizes in the lower nanometer regime have been reached \cite{international_roadmap_committee_international_2015}. With this strong decrease in size, technological requirements on the lithography systems used to fabricate those chips in mass production grew.

A basic principle of optical resolution known as the \emph{Rayleigh criterion} states, that the minimum structure size achievable with an optical system is roughly proportional to the wavelength used \cite{lord_rayleigh_xxxi._1879}. Consequently, while the first lithography systems used in the semiconductor industry operated in the visible spectrum, wavelengths had to be reduced down to the \gls{duv} at $\nm{193}$ used nowadays to keep pace with Moore's law. However, with feature sizes of only a few tenths of nanometer now necessary, a significant further reduction of the wavelength can not be avoided. The next-generation optical lithography uses wavelengths in the \gls{euv} spectral range of \nm{13.5}. This radiation is strongly absorbed by all materials, including air, challenging the design of the optical lithography systems by effectively ruling out any optical design based on transmission lenses for focusing and imaging of the structures, as those are unavailable. With the semiconductor industry at the verge of a major technological change, the topic of reflective optical elements for \gls{euv} radiation has gained a large amount of attention and experienced extensive research efforts \cite{bakshi_euv_2009}.

Back in 1972, Eberhart Spiller had proposed a new design for efficient mirror systems working at incidence angles near the surface normal. The idea was based on fabricating artificial layer systems reflecting portions of the incoming radiation at each interface that would interfere constructively at acceptable absorption levels overcoming the extremely low reflection otherwise seen from single surfaces \cite{spiller_low-loss_1972}. The result are bandwidth limited multilayer Bragg reflectors which fulfill the Bragg condition for constructive interference only for specific pairs of wavelength and angle of incidence and thus require specific design. At angles close to the surface normal, layers with thicknesses in the order of half the wavelength are necessary, which requires fabrication methods capable of precisely depositing layers of only several nanometers. Since the original proposal, multilayer systems have been realized using evaporation and sputtering techniques and demonstrated to increase reflection \cite{spiller_reflective_1976, underwood_layered_1981}. As the technology developed and more advanced sputtering techniques became available to fabricate at the necessary precision \cite{stearns_fabrication_1991}, the first important applications of focusing multilayer mirrors were space probes used for the observation of the sun in the \gls{euv} spectrum \cite{chauvineau_description_1992, clette_eit:_1995, spiller_soft_1994}.

Theoretical models and calculations of candidate systems for large reflectivity close to normal incidence at a wavelength of \nm{13.5}, show peak values of approximately $\SI{72}{\percent}$ by using multilayer systems based on molybdenum (Mo) and silicon (Si) \cite{barbee_jr._multi-spectral_1991,finkenthal_near_1990, barbee_molybdenum-silicon_1985}. State of the art systems fabricated today reach values closely above $\SI{70}{\percent}$ \cite{martinez-galarce_high_2000,bajt_improved_2002,braun_grenzflachen-optimierte_2003, braun_mo/si_2002, feigl_euv_2006}, which is still a few percentage points below the theoretical limit. This is of particular concern for the usage in \gls{euv} lithography systems, where $6$ to $9$ reflections are required to image a structure onto a wafer \cite{kaiser_euvl_2008, wagner_euv_2010}. The small difference to the theoretical reflection limit thus has a large impact on the total radiant power at the wafer level and is a very crucial point in the development of the next-generation lithography using \gls{euv} radiation. 

While the semiconductor industry undoubtedly is a very strong driving force in the development of \gls{euv} multilayer optics for \nm{13.5} wavelength, mirrors for other spectral ranges suffer from the same issue. A system relevant to this work is a mirror designed to reflect radiation in the range of the so-called \emph{water window} which is found between $2.3$ nm and $4.4$ nm. The water window is of special interest, because radiation in this spectral range shows low absorption in water, while it is absorbed by many elements naturally occurring in organic molecules such as proteins \cite{kirz_soft_1995}. This allows the study of biological systems in their native environment (water), where many proteins are biologically active. With the ability to produce radiation at those wavelengths at \gls{fel} sources \cite{ackermann_operation_2007, schreiber_first_2011}, more applications with strong and coherent pulses come in reach. High resolution imaging of protein samples, in addition to requiring the short wavelength, needs sufficient reflected radiation intensity and, more generally, optical elements capable of focusing and magnification. This can be achieved with high reflectance multilayer mirrors \cite{hertz_normal-incidence_1999,legall_compact_2012}. A candidate system relevant in this range is made from chromium (Cr) and scandium (Sc) applying the very same principle as introduced above, however, having much thinner layer thicknesses due to the shorter wavelength. While the theoretical reflectance limit at \nm{3.1} wavelength is calculated to reach values above $\SI{50}{\percent}$ \cite{schafers_cr/sc_1998}, state of the art samples only show reflectivities below \SI{20}{\percent} \cite{eriksson_14.5_2003, yulin_high-performance_2004}, less than half of the theoretically possible values.

The main reasons for radiation loss beyond unavoidable absorption inside the materials of both the Mo/Si and Cr/Sc multilayer systems are imperfections at the interfaces, such as compound formation, intermixing and roughness. As a result, the perfect multilayer system is distorted, since the interfaces are not chemically abrupt anymore. Thus, intermixing and compound formation lead to a diminished optical contrast and consequently to lower reflectance at the respective interface \cite{nakajima_interdiffusion_1988}. This is a known problem for multilayer optics, and measures to counteract this effect are the introduction of barrier layers hindering the formation of intermixing layers in some of the systems \cite{braun_grenzflachen-optimierte_2003,braun_mo/si_2002}. In the case of roughness, the result of reduced optical contrast at the interfaces is the same on average for the impinging wavefield, however, in this case with the addition of diffuse scattering outside the specular beam direction \cite{sinha_x-ray_1994}, which is not present in the case of pure intermixing.

To minimize interface distortions and to ultimately increase the reflectivity of the respective systems, the research and industry groups concerned with fabricating multilayer mirrors require detailed information on the actual structural properties and the interface morphology of their samples. The characterization of those multilayer systems is thus a cornerstone in the effort for improvement and the fundamental understanding of the effects involved. As the German \gls{nmi}, the \gls{ptb} is dedicated to precise measurements related to all fields of physics and technology providing metrology as its core mission. In fact, the international metrology organization, the Bureau International des Poids et Mesures, defines\footnote{Source: \url{http://www.bipm.org/en/worldwide-metrology/}} metrology as \emph{``the science of measurement, embracing both experimental and theoretical determinations at any level of uncertainty in any field of science and technology.''}. In this spirit, this thesis seeks to provide metrology for the important field of multilayer optics.

There are several characterization techniques that exist and have been applied to assess and quantify roughness and intermixing of materials at the interfaces of multilayer mirrors in the past. Some widely used example is \gls{tem}, which establishes a microscopic approach to the problem of assessing the interface morphology with resolution at the nanoscale \cite{stearns_thermally_1990, bajt_investigation_2001}. By imaging the layer stack, interface imperfections are directly visible. In combination with \gls{hreels}, element specific interface profiles can be deducted giving insight into the intermixing behavior of two (or more) materials at the interfaces \cite{egerton_electron_2011, prasciolu_thermal_2014}. A large downside of both methods, however, is the intrinsically local area of the image and thus the characterization of only very small local portions of the entire sample. Apart from that, the stack needs to be cut open to apply these techniques and thus leads to a destruction of the sample.

Another very popular method often used before and after deposition of a multilayer stack is \gls{afm} \cite{binnig_atomic_1986}. It is a scanning technique with nanometer resolution allowing to determine the morphology of a surface and thus investigate their roughness. However, it faces the same locality obstacle as \gls{tem} or \gls{hreels} and can only operate on exposed areas. Thus, the morphology of buried structures remains hidden from this method. Nevertheless, it is applied to determine the initial substrate roughness and the condition of the final top surface as an important prerequisite for high-quality multilayer mirror fabrication \cite{louis_progress_2000, bajt_investigation_2001}.

Apart from the direct and local scanning techniques, indirect ensemble methods based on the elastic scattering of radiation are accurate and extensively used in multilayer characterization. Those include \gls{xrr} and \gls{euv} reflectivity with \gls{reuv} as a variation of the latter. They are employed as standard approach in multilayer mirror fabrication and the subsequent characterization \cite{lim_fabrication_2001, bajt_investigation_2001, braun_mo/si_2002}. The major advantage is, that they are destruction-free, contact less and deliver information on the buried structure as well as on the top surface condition. Furthermore, statistical information across a large area depending on the beam footprint of the impinging radiation is obtained in contrast to the aforementioned local methods. However, it is no longer possible to directly gain information on the multilayer stack as theoretical models parameterizing its structural properties are required to calculate reflectivity curves and compare those to the measurement outcome. Reconstruction of the model parameters by fitting calculations to the experimental data raises the question of uniqueness and accuracy of the solution found. Even the applicability of the model and its limitations itself are of importance in these considerations. Several researches have shown, that the combination of \gls{euv} and \gls{xrr} can lead to significant improvements in the accuracy compared to standalone measurements with each technique individually \cite{yakunin_combined_2014}. \Gls{xrf} is another technique based on detecting fluorescence radiation of the materials inside the multilayer stack, that has been applied to add further complementary information to assist in the solution of this problem \cite{kortright_standing_1987, kawamura_interface_1994, ghose_x-ray_2001}.

While structural information on the layer stack can be obtained through reconstruction of a model by conducting these experiments, only limited information is gained on the roughness of the interfaces which are difficult to distinguish from intermixing. However, as only roughness causes diffuse scattering, the analysis of this off-specular intensity upon irradiation of a multilayer stack is a natural tool for characterization of the interface morphology. A lot of theoretical and experimental work has been conducted in relation to the study of diffuse scattering from multilayer samples, mostly in grazing incidence geometries using x-rays \cite{mikulik_x-ray_1997, sinha_x-ray_1994, de_boer_x-ray_1995, de_boer_x-ray_1996}, but also in the optical and \gls{euv} regime \cite{amra_light_1993, amra_light_1994, elson_light_1980, elson_relationship_1983, schroder_angle-resolved_2011, schroder_spectral_2014}, to deduct the desired information on the interface roughness.

This thesis is dedicated to the accurate and complete characterization of the structural properties and the interface morphology of multilayer mirrors based on the combination of several of the aforementioned methods. In the \gls{ptb} laboratories at the \gls{mls} and the \gls{bessy}, radiation in the spectral range from the terahertz up to the x-ray regime is available at several dedicated beamlines. Experiments implementing \gls{euv} and \gls{reuv} reflectivity, \gls{euv} diffuse scattering as well as \gls{xrf} have been conducted at the respective specialized end stations. External \gls{xrr} data was incorporated into the analysis to further strengthen the findings made. Experimental uncertainties inevitably associated with any measurement and model uncertainties are investigated with respect to each method employed. Based on different theoretical optimization algorithms, confidence intervals for each reconstructed parameter of the underlying models are determined as a consequence. As the layer thicknesses enter the sub-nanometer regime for Cr/Sc multilayer mirrors designed for the water window regime, a new model is proposed and evaluated. As a result, all of the methods are assessed with respect to their applicability for the structural investigation of such samples. Based on the reconstruction, the interface morphology is studied using \gls{euv} diffuse scattering analyzed based on the theoretical framework provided by the \gls{dwba} \cite{holy_nonspecular_1994, holy_x-ray_1993}. Again, the resulting parameter values are evaluated with respect to their validity by determining their confidence intervals based on the available data.

This work is structured based on the classification of the characterization methods rather than the samples under investigation. Chapter~\ref{ch_theo} introduces the fundamental theoretical concepts underlying the interaction of multilayer systems with \gls{euv} and x-ray radiation. The theoretical basis of the analytic experiments (\gls{euv} reflectivity, \gls{reuv}, \gls{xrr}, \gls{xrf} and \gls{euv} diffuse scattering) conduced in this thesis to characterize the various samples is given. In chapter~\ref{ch_exp}, the different experimental setups at the two storage rings \gls{mls} and \gls{bessy} employed in obtaining the data analyzed here are presented. In this work samples fitting in two major categories of multilayer mirrors for two different spectral ranges were investigated. Their fabrication was conduced using a sputtering technique, which is briefly reviewed. Furthermore, the extensive software that was developed over the course of this thesis is summarized there. The structural reconstruction of the Mo/Si and Cr/Sc multilayer mirrors based on the combination of the different experiments is presented in chapter~\ref{ch_spec}. Here, the validity of the models and the accuracy of the reconstructed parameters with their confidence intervals is discussed in depth. Finally, chapter~\ref{ch_diff} addresses the evaluation of the interface morphology of the very same samples based on the \gls{euv} diffuse scattering measurements and the models reconstructed in the previous chapter. The summary and conclusion of this thesis can be found in the last chapter~\ref{ch_summary}. Most of the work reported on here has been published in peer-reviewed journals \cite{haase_role_2014, haase_multiparameter_2016, haase_interface_2017}. At the end of each major section within the respective chapters, the corresponding publication related to the results presented is specified in detail.
