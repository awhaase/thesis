\section*{Abstract}

\thispagestyle{empty}

    Multilayer mirrors for the \glsdesc{euv} spectral range are essential optical elements of next-generation lithography systems and in scientific applications, e.g.~water window microscopes. Their lack in reaching theoretically predicted peak reflectivity values significantly hinders their applicability and raises the question of the reasons behind that limited performance. This thesis employs a combination of indirect characterization techniques using EUV and X-ray radiation to enable unambiguous judgments on the structural properties and interface morphologies of multilayer systems and to answer this question.
    
    This approach is used to study two sets of unpolished and interface polished Mo/Si/C multilayer systems designed to reflect EUV radiation with \nm{13.5} wavelength. They are fabricated with increasing molybdenum thickness from sample to sample. By examining the combination of \glsdesc{euv} reflectivity and \glsdesc{xrr} considering experimental uncertainties, structural parameters can be reconstructed and validated by deducting confidence intervals. Through the addition of the analysis of \glsdesc{euv} diffuse scattering, an observed dip in the peak reflectance for some samples can be related to variations in layer thickness and interface roughness associated with crystallization in the molybdenum layers. Increased roughness for samples at the crystallization threshold and intermixing can be identified impeding the measured reflectance.
    
    Furthermore, this methodology is applied to Cr/Sc multilayer mirrors for the water window spectral range having individual layer thicknesses in the sub-nanometer regime. The combination of the analysis of \glsdesc{euv} reflectivity and of \glsdesc{xrr} based on a binary layer model is shown to be insufficient to describe this system. The model is extended to explicitly take into account gradual interface profiles and strong intermixing. It is demonstrated by structural characterization and systematic validation of the extended model parameters, based on the analysis of \glsdesc{euv} reflectivity, \glsdesc{reuv} reflectivity, \glsdesc{xrr} and \glsdesc{xrf} experiments, that only the combination of those analytic methods yields a consistent result. By augmenting the characterization through the analysis of \glsdesc{euv} diffuse scattering, the cause for the low reflectivity is determined.

\cleardoublepage

\thispagestyle{empty}
\selectlanguage{ngerman}

\section*{Zusammenfassung}
    
    Deutsch

\selectlanguage{UKenglish}
\cleardoublepage
